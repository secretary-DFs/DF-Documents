% !TEX TS-program = xelatex
% !TEX encoding = UTF-8

% This is a simple template for a XeLaTeX document using the "article" class,
% with the fontspec package to easily select fonts.

\documentclass[a4paper, 11pt]{article} % use larger type; default would be 10pt

\usepackage[british]{babel}
\usepackage{fontspec}
\defaultfontfeatures{Mapping=tex-text}
\setmainfont[Ligatures=TeX]{Open Sans}

\usepackage[scale=0.9]{geometry}
\usepackage[parfill]{parskip} % Activate to begin paragraphs with an empty line rather than an indent
\usepackage[hidelinks]{hyperref}

\usepackage{graphicx} % support the \includegraphics command and options

\usepackage{multirow}

\title{Althing 2013 Minutes (Draft)}
\author{The DF Movement}
\date{\today}

\setcounter{tocdepth}{2}

\renewcommand{\arraystretch}{1.2}

\begin{document}
\maketitle

\tableofcontents

\section{Attendance}

In attendance were:
Pearl Ahrens,
Lily Bowler,
Joe Bowler,
Luke Breadmore,
Henry Care,
Holly Carter-Rich,
Adeline Childs (Adi),
Benjamin Cockerill-Evans (Ben),
Emily Connor,
Nic Craven,
Louise Delmege,
Ellie Ferguson,
Nick FitzGerald,
Joseph Flannagan (Joe),
Gabriel Hawkins-Pottier,
Ryan Hilson,
Sophie Holden,
Josh Hope-Collins,
Jay Imbery,
Leia Kennedy,
Emily Locke,
Joe MacMahon,
Lily MacTaggart,
Ruori McIntyre,
Ruth Mestel,
Eleanor Mestel (Ellie),
Alec Mezzetti,
Tes Monaghan,
Saskia Neibig,
Ruairi O'Boyle,
Ruth O'Sullivan,
Rosie Pearce,
Hannah Penwarden,
Jessica Poyner (Jess),
Esther Price,
Holly Purves,
Anna Rathbone,
Esther Rathbone,
Hal Ryan-Gill,
William Searby (Will),
Sophie Slater,
Imogen Smith,
Josie Tothill,
Ocea Weir,
Naomi Wilkins,
Angus Wood,
and Jack Yeo.

Additionally, Tom Brooks from General Council was observing and speaking with permission, particularly in relation to \nameref{motion:late1}.

\section{Mock Election}
To familiarise people with election procedure and consensus discussion, a mock election was held.  A role was decided on, nominations collected and a person elected to fill the role.

\subsection{Potential Roles}
\begin{description}
\item[Verbal Enforcer] to chastise Joe MacMahon every time he mentions the word `cray'.
\item[Wink Murderer] to periodically murder people throughout the weekend by the medium of the wink.
\item[Trivial Fact Person] to periodically, and on request, tell the discussion a trivial fact about a certain subject.
\end{description}

Wink Murderer was chosen.

\subsection{Nominations for Wink Murderer}
\begin{itemize}
\item Josie Tothill
\item Josh Hope-Collins
\item Anna Rathbone
\item Gabriel Hawkins-Pottier (withdrew)
\item Lily MacTaggart
\end{itemize}

Lily was elected.

\section{Motions}
\subsection{Motion 1 (Democratic Procedures)}
This DF Althing accepts the document entitled `DF Democratic Procedures', dated 5th September 2013, as an official guide to how business should be undertaken at business events (including Althings, Things, and mini-Things).  Relevant sections of the constitution will be modified to refer to this document.

(Joe MacMahon)

\subsubsection{Outcome}
Motion passes unanimously.

\subsection{Motion 2 (Dissolving Fundraising Rep)}

This Althing dissolves the role of Fundraising Rep and proposes a training session on fundraising take place at Old/New so that all Committee Reps are familiar with the basics.

\subsubsection{Rationale}

The nature of fundraising is currently such that the person
writing an application needs to have an intimate knowledge of what the
money is intended for.  People in the role of Affiliations, Campaigns,
Training, Projects and subsidiary positions thereof are better placed to
source and find funding related to their field, as well as being more
able to explain the need for the money and the value of the investment.  
The likelihood of successfully engaging someone in such a lengthy and
heavily administrative task is also increased when they are directly
involved in the outcome.

(Saskia Neibig)

\subsubsection{Amendment 1}

Append "The Constitution will be amended accordingly."

(Joe Bowler)

\subsubsection{Amendment 2}

It's the Training Rep's responsibility to ensure fundraising competencies across the movement.

(Josh Hope-Collins)

\subsubsection{Outcome}
Amendments 1 and 2 accepted by Saskia.

Motion clearly passes.

\subsection{Motion 3 (Chair of Committee)}

[Note, this motion was left on the table from Althing 2012, and as such some of the wording may need to be changed, e.g. "this year" means "the year 2011-2012".]

This Althing recognises that the role of Chair of DF Committee is a large one that cannot be properly completed by a Committee member already holding another role.  From this Althing onwards it shall become a primary role, albeit still a one-year role.  Those who wish to stand for the role shall be approved by Althing. Any of the people standing who come above RON in an Althing election shall be passed onto Committee.  At Althing, in the closed Committee meeting, a new Chair of DF Committee shall be elected from those who were approved by the rest of the Movement.  The constitution shall be amended accordingly.

\subsubsection{Rationale}

This year I (Louise Delmege) have been unable to fully fulfil all that I wished to as both comms and chair. I have found it impossible to keep the website up to date whilst still facilitating ctte, safeguarding DFs and acting as a laymember for those ctte members that neglect their roles at important times.

DF Camp is an example of this, whereby I was unable to keep the website updated every other day whilst acting as support for the DF Camp coordinating team during the time when the events rep and two of the coordinators were busy. This meant that many essential tasks went undone, this caused many, some severe, problems at the event. Coordinators and Ctte members should report when they will be busy and find laymembers to support them, when this and other responsibilities are ignored, it is up to the Chair to step in.
When Committee members fail in their responsibilities it is the Chair's responsibility to take over their roles. This is not possible while the Chair has another role. Making the Chair of DF Committee a primary role will ensure that, in future, this can be avoided. It will also allow more time for the Chair to check up on Committee members and support them before it gets to the point where vital parts of their role go undone.

The complex election procedure ensures that, since the Chair of DF Committee is a member of DF Committee and carries all the responsibilities of a ctte member, they are approved by the Movement, from the floor of Althing. But, the choice of chair is still made by the upcoming DF Committee after discussion with the current Committee (Changeover being Old/New). Since the Chair of DF Committee is the facilitator of ctte is is vital that it is they who choose them. This will also give the role of Chair the same changeover time as all other roles.

(Louise Delmege, seconded by Ruori MacIntyre)

\subsubsection{Outcome}

Amendment 1\footnote{Text of amendment lost, proposed by Zoë Fidler.} not accepted and clearly falls.

Amendment 2\footnote{Text of amendment lost, proposed by Josh Hope-Collins.} not accepted and clearly falls.

Motion clearly passes.

\subsection{Motion 4 (Shadow MEST-UP)}

This Althing creates the role of Shadow MEST-UP Co-ordinator.  This shall be a non-committee role, and last for one year, like the role of MEST-UP Co-ordinator itself.  This Althing also mandates that only previous Shadow MEST-UP Co-ordinators or MEST-UP reps may stand for the role of MEST-UP Co-ordinator, and Shadow MEST-UP Co-ordinators would not be forced to stand for the role of MEST-UP Co-ordinator.

\subsubsection{Rationale}

MEST-UP is a vital part of our organisation and, as such, the running of it is a very important job. Issy is currently doing this job brilliantly, but I am aware that the handover was not as smooth as it could have been, and a lot of information that was available to previous MEST-UP coordinators has had to be found from scratch, as well as plans for workshops and for training the MEST-UP team. As such, I feel that it is important that there be a shadow MEST-UP coordinator, both so that they can learn about how to properly fulfil this role and so that information can be retained during the transition between MEST-UP coordinators to help MEST-UP become an even better project/programme than it already is.

(Naomi Wilkins)

\subsubsection{Amendment 1}

Replace `Shadow' with `Vice'.

(Lily Bowler)

\subsubsection{Amendment 2}

The Vice Chair would be elected in a similar way to Vice Chair of DF Committee from the pool of MEST-UP reps at the first MEST-UP meeting of the year.

(Josh Hope-Collins)

\subsubsection{Outcome}

Amendments 1 and 2 accepted by Naomi.

Motion falls with 7 for, 14 against, 14 abstentions\footnote{7+14 = 21 < 27 so not quorate anyway.}.

\subsection{Financial Report}
[Insert Financial Report here.]

\subsection{Motion 5 (Venturer Committee Donation)}

This Althing recognises the work that Venturer Committee Liaison does for DFs and that the source of funding that Venturer Committee previously had no longer exists.  Therefore it resolves to pay for Venturer Committee Liaison to attend Venturer Committee meetings.  This is an annual cost and will be in the region of £200, additionally £300 will be given to Venturer Committee to support them financially with the goal of making Venturer Committee sustainable.  The £300 payment will set a precedent for future Althings to recur at their discretion, while the c. £200 expenses will be an automatically annually recurring payment.

A report will be submitted to the DF Movement by Venturer Committee of how the money will be/has been spent.

\subsubsection{Rationale}
In order for Venturer Committee to become a sustainable resource for Woodcraft Folk, regions are being asked a yearly contribution of £500 to cover costs of Venturer Committee meetings.  While DFs is not a region of Woodcraft, we benefit from Venturer Committee's projects and events, better Venturer to DF transition, and more confident new DFs to stand for roles within DFs.

(Sophie Holden)

\subsubsection{Amendment 1}

Delete `additionally £300 will be given to Venturer Committee to support them financially with the goal of making Venturer Committee sustainable'.

(Louise Delmege)

\subsubsection{Outcome}

Amendment 1 not accepted and clearly falls.

Several amendments were proposed and accepted, and edited inline.

Motion clearly passes.

\subsection{Motion 9 (Support for Syria)}

In light of the current situation in Syria, this DF Althing voices its opposition to British military intervention.  It also instructs the International Opportunities Representative to build ties with organisations similar to ours within Syria so that we can offer what help we are able to.

This Althing also instructs the Treasurer to make a donation from our reserves of £500 to the International Committee of the Red Cross, and £500 to the International Federation of Red Cross and Red Crescent Societies.  Financial help is not the only help these organisations require, and as such this DF Althing encourages DFs to volunteer if they feel able.

\subsubsection{Rationale}

Woodcraft is an organisation which supports and advocates for peace and co-operation over war and violence.  We voiced our opposition to the war in Iraq in 2003 and we should voice our opposition now.  Intervention in Syria will only make things worse: as we have seen with Iraq and Afghanistan, countries built on western intervention lack stability and extremism flourishes.  Intervention is a short term solution to a problem which needs to be resolved by the Syrian people.  Moreover, many rebel factions within Syria are against intervention for these same reasons.

We should also recognise, however, that in the current political and economic system, we as individuals and collectively as an organisation are powerless to influence the leaders of this country, and should therefore try and mitigate the disaster which is eventually coming.  We should give as much help as we can to those people -- the Syrian people -- who really need it.  Tangentially, we should also be on guard in future for propagandised claims by our own government that there is no money to sustain public services, when we know that money wasted on needless wars and sustaining an imperialist military presence could be spent supporting the people of this country who are in need.

(Joe MacMahon)

\subsubsection{Amendment 1}

Since we are legally prohibited from making a donation to the Red Cross, replace the last paragraph with the following:

\begin{quote}
This Althing also encourages DFs to volunteer with Doctors Without Borders, the International Committee of the Red Cross or the International Federation of Red Cross and Red Crescent Societies if they feel able, and additionally to support them financially if they feel able.
\end{quote}

(Joe MacMahon)

\subsubsection{Outcome}

Amendment 1 accepted by Joe.

Motion passes unopposed.

\subsection{Motion 10 (Non-Committee Expenses 1)}

This Althing instructs that non committee members of DFs who are organising or coordinating national DF projects/events should be expensed to Things on a case by case basis to more formally discuss their plans. 

\subsubsection{Rationale}

It would be useful and positive to expense active DFs who are organising and coordinating things/projects/events to attend Things. (I have the members of the early stages of The Share in mind, it would have been useful to have Sam Sender who had a lot of big ideas and plans to be able to attend Things and discuss those ideas formally.  Other examples of when this could have been useful for the movement include when I (Imogen Smith) was coordinating the `Travelling DF centre' for CoCamp or the Welsh Valentines hostel weekend event.  I am absolutely sure that there will be plenty more similar examples of DFs organising things off their own back to come in the future.)  This motion would allow committee and the DF movement to more effectively discuss what is actually being organised and planned on a national level within the DF movement rather than just what is happening within committee.  This motion used alongside the existing Project Fund Application grant would also give active DFs the recognition and support they deserve to run and coordinate exciting national projects and events.  It would also promote participation in national DFs and empower DFs to make their ideas happen regardless of whether or not they hold a committee role.  Things would be made more relevant, effective and inclusive.

(Imogen Smith)

\subsubsection{Outcome}

Motion withdrawn.

\subsection{Motion 11}

\subsubsection{Amendment}

Three expensed places to Things shall be created on top of the Committee places and Two event coordinator places.  These roles are for volunteers not in a specific, expensed role, who are taking part in a short or long term project of value to the Movement, e.g. a local event or part of the campaign.  Applications can be made for these places to DF committee before the event.  No application form is needed, only an email to the Chair of DF Committee explaining why the Movement would benefit from giving them a place.

\paragraph{Rationale}
Committee members often don't know until quite late whether or not they can come to a Thing.  If they can't, Regional Council have first priority for places.  This means applying for their unused places is difficult. Having three certain places from the start will make this process much easier.

(Louise Delmege)

\subsubsection{Outcome}

Amendments 1, 2 and 4 accepted by Saskia.

Amendment 2 not accepted.  Vote count: 18 for, 3 against, 10 abstentions.  Amendment falls due to not being quorate\footnote{18+3 = 21 < 29}.

Motion passes unanimously.

\subsection{Procedural Point Regarding Motions 6 and 7}

It was decided to choose an ordering of the two motions and discuss them sequentially\footnote{It should be noted that this is against precedent and codified procedure; resulting action should be clarified and whether the Democratic Procedures need to be amended or nullified accordingly.}.

It was decided to discuss Motion 6 and then Motion 7, however if Motion 6 passed then Motion 7 would become irrelevant and would therefore not be discussed.

\subsection{Motion 6 (Dissolution of Regional Council)}
\label{motion:rcdissolve}

This DF Althing calls for the dissolution of Regional Council and the creation of a Districts rep on DF Committee in place of the Chair of Regional Council.

\subsubsection{Rationale}

The DF Movement works successfully on a UK-wide and a district level.  We do not work at the level of regions and nations within the UK.  There have been no successful regional/national events in recent years, bar the Welsh valentines Hostel, but, Wales being comprised of 3 districts, this should truly be considered a district level event.  A Districts rep would be responsible for encouraging and supporting similar district events and projects.

None of the members of Regional Council have fulfilled their mandated role for the last few years.  Some members have worked towards their roles but found it impossible to coordinate DFs at a regional/national level.  A single Districts rep could do the mandated work of supporting the creation of new districts more successfully as they would be easier to contact and could collect more experience.

We struggle to fill all the places on Regional Council and have been unable to do so in recent years.  There is not enough enthusiasm for this council to work.  A single Districts Rep would be more effective.  They could do more than is currently done by all Regional Council members.

The existence of Regional Council may actually be preventing others from running events.  Being told that it is an elected role to do so can be off putting to people without said role.  Also, unused Committee places at Things can currently be used by Regional Council members; this is of little to no benefit to DFs.  These places could instead go to valuable volunteers doing short, or long term projects.

RC members have a vote at AG.  This vote is rarely, if at all, used by the member, it is most often taken by another DF of that region/nation.  There is no reason this cannot continue without having to first track down the often hard to contact RC member and first ask them if they will use the vote.

Regional Council do not help the DF Movement.  Dissolving Regional Council will allow us to fund active volunteers to Things.  A Districts rep to Committee will increase DFs district level support more cheaply, and effectively.

(Louise Delmege)

\subsubsection{Amendment 1}

The responsibility of organising coordination for each Thing falls to the Districts Rep.  The coordinator should ideally be from the region of the Thing.

(Joe Bowler)

\subsubsection{Amendment 2}

An `Other' role called London Region Liaison should be created, who should report to the Districts Rep.

(Gabriel Hawkins-Pottier)

\subsubsection{Amendment 3}

Have elected people in each region to have an online presence and encourage DFs in that Region to go to events.

(Gabriel Hawkins-Pottier)

\subsubsection{Amendment 4}

Do a thingie for Scotland like the thingie for London.

(Gabriel Hawkins-Pottier)

\subsubsection{Outcome}

Amendments 1 and 2 accepted by Louise.

Amendment 3 rejected and clearly falls.

Amendment 4 rejected and clearly falls.

Motion clearly passes.

\subsection{Motion 7 (Supporting Regional Council)}

This Althing recognises that:
\begin{itemize}
\item Regional Council have been less effective recently
\item communication between Regional Council has been less effective
\item we are finding it hard to fill all Regional Council places
\end{itemize}

This Althing resolves to set aside £2275 for Regional Council to be able to meet three times a year and become more useful again.  This will be applied for rather than come out of the central budget.

\subsubsection{Rationale}

This £2275 would go towards funding 3 times in which Regional Council can meet at:
\begin{itemize}
\item the first DF Thing after Althing
\item the middle Venturer Committee meeting
\item a separate event purely for Regional Council to meet
\end{itemize}

\paragraph{DF Thing}
\begin{center}
	\begin{tabular}[H]{l | r r r r r r}
				& Venue	& Food	& Travel	& Stationery	& Other		& Total \\ \hline
	Per person	& £20	& £6		& £35	& £1			& \textendash & £62 \\
	Total (10)		& £200	& £60	& £350	& £10		& \textendash & £620
	\end{tabular}
\end{center}

\paragraph{Venturer Committee Meeting}
\begin{center}
	\begin{tabular}[H]{l | r r r r r r}
				& Venue	& Food	& Travel	& Stationery	& Other		& Total \\ \hline
	Per person	& £20	& £6		& £35	& £1			& \textendash & £62 \\
	Total (10)		& £200	& £60	& £350	& £10		& \textendash & £620
	\end{tabular}
\end{center}

\paragraph{Separate RC Meeting}
\begin{center}
	\begin{tabular}[H]{l | r r r r r r}
				& Venue	& Food	& Travel	& Stationery	& Other	& Total \\ \hline
	Per person	& £10	& £6		& £35	& £2.50		& £10	& £63.50 \\
	Total (10)		& £100	& £60	& £350	& £10		& £100	& £635
	\end{tabular}
\end{center}
For this venue we could use someone's house and offer the residents £100 compensation.

\paragraph{Final Breakdown}
\begin{center}
	\begin{tabular}[H]{l | r}
	DF Thing					& £620 \\
	Venturer Committee Meeting	& £620 \\
	Separate RC Meeting		& £635 \\
	RC Projects				& £200 \\
	Contingency				& £200 \\ \hline
	Total						& £2275
	\end{tabular}
\end{center}

(Sophe Holden)

\subsubsection{Outcome}
Motion falls automatically as \nameref{motion:rcdissolve} passed.

\subsection{Late Motion 3 (Share Rep)}
This Althing dissolves the role of Share Coordinator and mandates that their duties to be delegated to the Projects Rep.  If the Projects Rep wishes to appointoinette a Share Rep during their term, they may do so at their discretion.

\subsubsection{Rationale}
The Share project currently is not working in a capacity to warrant a completely separate role.

(Lily Bowler)

\subsubsection{Outcome}

Motion passes unopposed.

\subsection{Late Motion 4 (MEST-UP)}
This Althing moves that MEST-UP Rep become a Committee role.  To clarify, this would mean it is expensed to Things and is a 2 year role.  The constitution should be amended accordingly.

\subsubsection{Rationale}
MEST-UP rep is a really big role.  It requires more support than is currently available to them.  Committee provides Lay Members, Vice Chair, Chair and great connections to skilled supporting roles.  Committee already deal with confidential issues.  Chair and Vice Chair are trained in this area through their safeguarding training.

(Louise Delmege)

\subsubsection{Outcome}

Motion clearly passes.

\subsection{Procedural Point Regarding Late Motion 5}

It was decided not to bump Late Motion 5 up the agenda, and leave it for discussion after elections.

\subsection{Late Motion 1}
\label{motion:late1}
This Althing agrees that DFs give Woodcraft core funds £4,000
towards the fee that Woodcraft Folk pays to be a member of IFM-SEI (our
international umbrella organisation).

\subsubsection{Rationale}
Woodcraft Folk core funds are very low at the moment and
Woodcraft nationally has been running at a loss for a number of years.
The international element of our work is important to all age groups
within the organisation, including DFs who benefit from shared
experiences, opportunities and exchanges. By specifically putting
funding towards this area of work, the DF movement will be ensuring that
Woodcraft Folk continues to show solidarity with other like minded youth
groups around the world, continues to be a member of a IFM-SEI and
continues this great work, including hosting over 40 Catalans at VCamp,
sending delegates to the International Volunteers Camp in Palestine and
to Queer Easter, linking with Brick Kiln workers in Pakistan and sending
messages of support to Hungarian partner organisations who are attacked
for protesting against fascism in Budapest.

(General Council, via Louise Delmege)

\subsubsection{Amendment 1}

Append ``This is conditional on Woodcraft Folk remaining in IFM-SEI.''.

(WIll Searby)

\subsubsection{Amendment 2}

This would not come out of the annual budget, so that we don't make an annual budget loss.

(Lily Bowler)

\subsubsection{Amendment 3}
Append the following:

\begin{quote}
We would like to say the following to General Council:
	\begin{quotation}
	In addition to this money we would like to continue and increase our support of Woodcraft through our members and our skills in the following ways:
	\begin{itemize}
	\item Visa campaign
	\item Friends of the Folk
	\item Grand Committees
	\end{itemize}
	
	The money is also in addition to the Roots Fund, our investment in Venturer Camp and Venturer Committee.

	We hope that this will continue to foster friendliness, comradeship and mutual respect between General Council and DFs.''
	\end{quotation}
\end{quote}

(Joe MacMahon)

\subsubsection{Outcome}

Jess Poyner declared a conflict of interest: Her step-dad is employed by Woodcraft.  She is chairing but is happy to step down and leave the room if necessary.

Amendments 1, 2 and 3 accepted by Louise.

Motion passes unanimously.

\subsection{Late Motion 3 (DF Chair)}
This Althing recognises the trials of having the role of `DF Chair' and therefore resolves to make a model DF Chair 10cm tall.  This shall be made by Lily Bowler and be taken to all DF events and seen as the DF Chair, which we shall all look to and see as a figurehead.  In addition a DF Chair role will be elected who will wear the DF Chair as a badge.  If DFs acquire a van then this shall be made in life size and the DF Chair role should be to sit on the DF Chair.

(Lily Bowler)

\subsubsection{Amendment 1}

Change to 15cm.

(Sophie Holden)

\subsubsection{Amendment 2}

Change `trials' to `trials and tribulations'.

(Josh Hope-Collins)

\subsubsection{Amendment 3}

When Lily Bowler leaves DFs, the role of Chair Carer shall pass on to another Lily.

(Lily MacTaggart)

\subsubsection{Amendment 4}

The role of Chair Carer should be a secondary committee role, and the Constitution shall be amended accordingly.

(Saskia Neibig)

\subsubsection{Outcome}

Amendment 1 not accepted and clearly falls.

Amendment 2 accepted.

Amendment 3 not accepted and clearly falls.

Amendment 4 not accepted and passes. Vote count: 23 for, 4 against, some abstentions.

Motion clearly passes.

\subsection{Motion 8 (DF Library Funding)}

This DF Althing approves the PFA submitted for the DF Library.

\subsubsection{Rationale}

The PFA gives money to a member of DF Committee to develop a piece of software to manage the DF Library, and as such it cannot be approved by DF Committee itself.  Hence a good option would be to have it approved by Althing.

(Joe MacMahon)

\subsubsection{Amendment 1}
Append ``to the sum of  £890''.

(Saskia Neibig)

\subsubsection{Outcome}

Amendment 1 passes.

Motion clearly falls.

\subsection{Motion 12 (No Profit on Events)}

DF events should not be budgeted to create profit unless they are primarily fundraising events, also any profit made through going under budget on a DF event should be used to reduce the costs of future events, to increase participation. This excludes the five pounds for fairer fare.

\subsubsection{Rationale}
DFs is a non profit organisation, but, despite this, year on year the sum in our bank account gets larger.  I believe that we should aim to reduce this sum of money, as currently it is doing little sitting in the bank account.  This motion reduces the income received and instead reinvests it into events allowing them to be cheaper and therefore more accessible, increasing participation and making DFs better.

Some people might argue that this motion is unsustainable, as we could lose our Workers beer places and so some of our income and eventually we might need to increase the costs of events.  I believe that firstly this would take a considerable amount of time to happen and secondly that, if this motion starts to become unreasonable, future DFs will vote to change this.

Others may say that the money is already intended to fund stuff like Things, expenses, New Roots, or the Huge Ideas Pot.  Despite these outgoings DFs are still making a large gain.

(Adam Fidler)

\subsubsection{Amendment 1}

Remove `create profit' and replace with `create a profit of over 20\% of the final event budget excluding fairer fare'.

(Alec Mezzetti)

\subsubsection{Amendment 2}

Remove `also any profit made through going under budget on a DF event should be used to reduce the costs of future events, to increase participation'.

(Sophie Slater)

\subsubsection{Outcome}

Note: Adam Fidler is not in attendance so all amendments stand as not accepted.

Amendment 1 passes.

Amendment 2 passes unopposed.

Motion falls unsupported.

\section{Remaining Motions}

Due to time constraints, it was decided to stop business and delegate the power to pass the remaining motions to one of 4 events.  Motions could be either delegated to Old/New, to the mini-Thing at Winter Wonderland, to South-West Thing, or tabled until Althing 2014.

\begin{description}
\item[Motion 13] Old/New
\item[Motion 14] Winter Wonderland\footnote{Suggestion to incoming Committee: apply to next Thing anyway.}
\item[Motion 15] Winter Wonderland
\item[Motion 16] Withdrawn
\item[Late Motion 2] South West Thing
\item[Late Motion 5] Tabled for 2014
\item[Late Motion 6] South West Thing
\item[Late Motion 7] Tabled for 2014
\item[Late Motion 8] Old/New
\end{description}

\subsection{Motion 13 (Lapsing Policies)}

\subsubsection{Motions to be repealed}

This Althing repeals the following policy-making motions.  The reason for repealing is given alongside each motion, unless it is simply that it is no longer current practice.

\paragraph{From 2008}

\begin{description}
\item[Motion 2]  This DF Althing recognizes the difficulty of finding venues for winter wonderland and as such instructs that venues be contacted at least 15 months in advance and be booked at least 15 months in advance.

\item[Motion 20]  This DF Althing recommends that where possible, Current affairs \& controversial issues should be presented to DFs with both sides of the argument (i.e. ‘for’ and ‘against’). All speakers on events should allow time for an extended question and answer session.

\item[Motion 29]  This DF Althing instructs committee to pay expenses to the following only: All members of DF committee attending Things, Althing and Old New, Regional Developers to the Thing in their region, event organisers to the Thing before their event and all others approved by committee.

[Policy on expenses is going in the Democratic Procedures document.]

\item[Emergency Motion G]  This DF Althing recognises that since last years change to the constitution regional reps have been extremely ineffective and they should therefore be partiall expensed to come to things. This would ensure they actually do things for their regions.  Committee decides a block of money yearly at Old/New.
\end{description}

\paragraph{From 2009}

\begin{description}
\item[Motion 1]  This DF Althing recognizes that many of our procedures at Althing are based on tradition and as such formalizes them thus:
	\begin{enumerate}
	\item To pass a motion there must be a simple majority of those not abstaining.
	\item To pass a motion changing the constitution there must be a two thirds majority of those present.
	\item If the votes for and against a non-constitutional motion are equal then the motion shall fail to pass.
	\end{enumerate}

[This is now written down in the Democratic Procedures so this policy is unnecessary.]

\item[Motion 2]  This DF Althing institutes the following procedure for tied votes, first the number of first place votes shall be considered, the higher number wining, then second place, and so on, if it remains a tie we will facilitate a revote as quickly and democratically as possible and consult standing orders. This process is to be added to the constitution as appropriate.

[Same reason as above.]

\item[Motion 4]  This DF Althing instructs that the cost of all DF events should be affordable to everybody in both an immediate and holistic sense and that funding should be applied for in each case.

\item[Motion 9]  This DF Althing instructs that each region is to hold an annual social event, and that Regional developers are to ensure this happens (not necessarily organise it). This could be as large or small as the region calls for. Holding these would give greater focus to regional developers and help to make DFs more 'open and accessible to all' in the spirit of the wider movement.
\end{description}

\paragraph{From 2010}

\begin{description}
\item[Motion 10 (Part D)]  DF committee should attempt to create and maintain a list of Kinsfolk contact details which keeps track of relevant and useful skills.  The list could be used to invite Young Kinsfolk to events in order to conduct skill sharing session and workshops and also to do volunteer work for DFs.  This already happens in a fairly ad-hoc way, but a list would formalize it.  A list may be problematic initially, but will ensure that more information is retained over a long period of time.  DFs should not seek to rely upon the skills of Young Kinsfolk, but rather facilitate the passing on of these skills to DFs.  Be creative with what a skill is, it could be anything from an expansive knowledge of climate science to being really good at running wide games.

\item[Motion 13]  This DF Althing believes that motions are not always the most productive or inclusive way of
changing things within the movement. This DF Althing moves that Althing should provide a platform
at which problems and ideas can be discussed and solutions found in an open discussion. These
discussions should be focussed on coming up with positive actions to take forward. Further DFs will
bring a similar motion to Annual Gathering 2011 for all Woodcraft Folk.

[This is going in the Democratic Procedures so this policy is unnecessary.]

\item[Motion 13]  This Althing recognises that support for young people working in groups is important for DFs and Woodcraft folk and therefore instructs the training rep to organise leadership training at least one DF social event per year.

[While we have had leader training this year, it has not been part of a social event.]

\item[Motion 15]  This DF Althing instructs that Gift Aid be claimed on all eligible DF income to include donations made when attending events. The money raised through Gift Aid shall be considered restricted funding donated to the Koodoo Project for a national Woodcraft Folk campsite.
\end{description}

\subsubsection{Motions to be kept for another year}

This Althing does not repeal these motions, and as such they stand for another year.

\paragraph{From 2008}

\begin{description}
\item[Motion 6]  This DF Althing instructs first aid training to take place at a social event each year. The organisation of this should be taken on by the training rep.
\end{description}

\paragraph{From 2009}
\begin{description}
\item[Modified Motion 3]  This DF Althing realizes that the food served on DF events should be entirely inclusive. A vegan diet is healthy, cheap and, importantly, provides adequate nourishment for all DFs. Most importantly, a vegan diet has many environmental advantages (as fulfilling motion 11 of Althing 2009). KP’s should strive to provide food which is local to the camp, environmentally friendly, organic, from a cooperative source, fair trade and takes into account animal welfare. This means treating meat and other unethical foods as supplements rather than staples and allowing the possibility of meat free camps. Ultimately the decision is to be taken by KPs, who should endeavor to communicate their choices and actions to the wider movement.

\item[Emergency Motion I]  This DF Althing wishes to henceforth delegate the co-ordination of Span That World With Music to the events rep.
\end{description}

\paragraph{From 2010}
\begin{description}
\item[Motion 2]  This Althing instructs all event coordinators to programme a 'Mini Thing' into all social event
programmes as an educational event in which all willing DFs are introduced to the process of DF
business and where any business relevant to social events is discussed and decisions are made.

\item[Motion 3]  This DF Althing demands a National DF dress up day Span That World With Costume.
\end{description}

\paragraph{From 2011}
\begin{description}
\item[Motion 1]  This DF Althing instructs that the DF symbol can only be used for and by DFs and not by the wider Woodcraft movement. This DF Althing recognises that the DF logo should not be used in place of the Woodcraft Logo, but can be used alongside it, or with credit to the DF Movement.

\item[Motion 3]  This DF Althing instructs the incoming DF committee to create the `Group Development Fund.' This fund shall be an unlimited resource from which individuals and groups of DFs can apply for money. The money can be used under the following restrictions:
	\begin{itemize}
	\item The money must be used only to develop groups. In particular, new groups of Woodchips, Elfins, Pioneers and Venturers. 
	\item The fund is unlimited BUT is left under the discretion of DF committee to safeguard the financial well-being of District Fellows. Applications will be considered on a case-by-case basis.
	\end{itemize}

\item[Motion 4]  This DF Althing instructs event to make a statement regarding consent, along with Drugs and Alcohol policy.

\item[Motion 10]  This DF Althing proposes that people expensed to business events should have to pay for their food on events.
\end{description}

(Joe MacMahon)

\subsection{Motion 14 (Timing of Business)}

This Althing mandates that all DF business at Things (including
mini-Things) and Althings will take place between 8am and midnight.

Additionally, no business should take place over a mealtime.

(Zoë Fidler)

\subsection{Motion 15 (Safe Spaces for Women and Trans* People)}

This Althing mandates that every DF social event should have an area set aside to act as a safe space for women and trans* people.

\subsubsection{Rationale}

While Woodcraft is a much more welcoming place than most, safe spaces are still needed and it is important to have a separate safe space (even if it is just a curtained off part of the MEST-UP tent) for those who need it.

(Kitty Howse)

\subsection{Motion 16 (Venturer Committee and Venues)}

This Althing instructs that DF Regional Council should offer to share a venue with Venturer Committee at least once a year as part of beginning to build a better bond between Venturer Committee and Regional Council.

\subsubsection{Rationale}
The relationship between Venturer Committee and Regional Council helps with the transition to DFs and means that more new DFs will arrive confident and more likely to stand for roles within DFs.  Sharing a venue is a good way of getting to know each other without having to hold an evente exclusively for this purpose.  It is also a good way of supporting Venturer Committee by giving them a chance to talk with people doing a similar role.

(Sophie Holden)

\section{Late Motions}

\subsection{Late Motion 2 (Yearly Reports)}
This Althing recognises that non-committee roles are as important as committee roles and therefore mandates every role elected to give a yearly report.

\subsubsection{Rationale}
Why shouldn't they? It helps the following role to understand the trials and highlights of the roles.

(Lily Bowler)

\subsubsection{Amendment}
Also quarterly reports.

(Lily Bowler)

\subsection{Late Motion 5}

This Althing mandates that all primary roles except chair be elected from the floor of Althing for a term of one year.  The Constitution should be amended accordingly.

(Alec Mezzetti)

\subsection{Late Motion 6}

This Althing clarifies that an automatic vote of no confidence shall take place because of physical or mental incapacity only when the internal Committee support structures have failed.  The Constitution shall be amended accordingly if necessary.

(Joe MacMahon)

\subsection{Late Motion 7}

This Althing notes the difficulties both General Council and DF Committee have faced in running the DF account in the first half of the last year.  We therefore resolve to replace section B.1.7.2.2 of the Constitution with the following:

\begin{quote}
If the Shadow Treasurer is unable to fully take on the role of Treasurer, an additional Shadow Treasurer is elected for one year.  General Council and DF Committee will support both Shadow Treasurers in their training for the year and will help write the yearly report for our financial audit.
\end{quote}

(Ellie Mestel, Joe Bowler)

Secretary's note: section B.1.7.2.2 currently reads, 
\begin{quote}
If the Shadow Treasurer is unable to fully take on the role of Treasurer, the General Secretary, Chair and Treasurer of the Woodcraft Folk will take on the Treasurer’s responsibilities until the Shadow Treasurer has had sufficient training.
\end{quote}

\subsection{Late Motion 8}

All signatories will have a shared `payments' email account, and Nick FitzGerald (incoming Shadow Treasurer) should look into accounting software.

(Ellie Mestel, Joe Bowler)

\section{Musings}

\emph{This section is not yet completed.}

\section{Election Results}

\begin{center}
	\begin{tabular}[H]{l || l | l}
	Role						& Nominees		& Elected \\ \hline \hline
	Treasurer					& Ellie Mestel		& Ellie Mestel \\ \hline
	\multirow{2}{*}{MEST-UP}	& Issy Rose		& \multirow{2}{*}{Issy Rose} \\
							& Alice Griggs		& \\ \hline
	Venturer Committee Liaison	& Lily MacTaggart	& Lily MacTaggart \\ \hline
	\multirow{2}{*}{Shadow Treasurer}	& Nick FitzGerald	& \multirow{2}{*}{Nick FitzGerald} \\
							& Zoë Fidler		& \\ \hline
	\multirow{2}{*}{Webfairy}	& David Moore		& \multirow{2}{*}{David Moore} \\
							& Luke Breadmore	& \\ \hline
	\multirow{2}{*}{Zine Editor}	& Alec Mezzetti	& \multirow{2}{*}{Alec Mezzetti} \\
							& Hufi Neibig		& \\ \hline
	\multirow{2}{*}{Districts}	& Sophie Holden	& \multirow{2}{*}{Emily Locke} \\
							& Emily Locke		& \\ \hline
	\multirow{2}{*}{Events}	& Imogen Smith	& \multirow{2}{*}{Lily Bowler} \\
							& Lily Bowler		& \\ \hline
	First Aid					& Angus Wood		& Angus Wood \\ \hline
	\multirow{2}{*}{Chair}		& Adi Childs		& \multirow{2}{*}{Adi Childs \& Jack Yeo} \\
							& Jack Yeo		& \\ \hline
	Campaigns				& Saskia Neibig	& Saskia Neibig \\ \hline
	\multirow{2}{*}{International Opportunities}	& Anna Rathbone	& \multirow{2}{*}{Anna Rathbone} \\
							& Leia Kennedy	& \\ \hline
	London Liaison				& Ruairí O'Boyle	& Ruairí O'Boyle \\ \hline
	\multirow{5}{*}{Lay Members}	& Ryan Hilson	& \multirow{5}{*}{Ryan Hilson \& Josh Hope-Collins} \\
							& Josh Hope-Collins	& \\
							& Anna Rathbone	& \\
							& Jack Yeo		& \\
							& Holly Purves		& \\ \hline
	General Council Reps		& Naomi Wilkins	& Naomi Wilkins \\ \hline
	\multirow{2}{*}{Projects}	& Joe Flannagan	& \multirow{2}{*}{Joe Flannagan} \\
							& Gabriel Hawkins-Pottier	& \\ \hline
	Training					& Naomi Wilkins	& Naomi Wilkins \\ \hline
	\multirow{2}{*}{Workers' Beer}	& Adi Childs	& \multirow{2}{*}{Toby Attrill} \\
							& Toby Attrill		& \\
	\end{tabular}
\end{center}

\pagebreak

\section{Wink Murder Deaths}
\begin{description}
\item[11.25am] Josh Hope-Collins
\item[12.00pm] Hal Ryan-Gill
\item[12.25pm] Joe Flanagan
\item[1.10pm] Henry Care
\item[2.00pm] Jack Yeo
\item[2.42pm] Tes Monaghan
\item[5.13pm] Pearl Ahrens
\item[6.00pm] Joe MacMahon
\item[6.11pm] Lily Bowler
\item[8.00pm] Nick FitzGerald
\item[10.29pm] Issy Rose
\item[12.00am] Angus Wood
\item[11.52am] Ruth Mestel
\end{description}

\section{Quotes}
``I'm really passionate about death.'' --- Josie Tothill

``This is perfectly legit! It's how all charities work.'' --- Saskia Neibig

``Bad idea.  Always say you're the Treasurer.'' --- Ellie Mestel

``Any motion is too much motion.'' --- Sophie Holden

``I want to, hopefully, put some people's minds at risk.'' --- Saskia Neibig

``It's not about the money.'' --- Ellie Mestel [DF (Shadow) Treasurer]

``If anyone wants to talk about funding, we can sit down over dinner and chat.'' --- Saskia Neibig

``I've been doing some agenda-bending; we're doing the Chair motion.'' --- Joe MacMahon

``Why are we doing this one before the erections?'' --- Jack Yeo

``I've done a lot of things in A5.'' --- Alec Mezzetti

``I'm used to having a lot of people unloading on me at once.'' --- Jack Yeo

``Just, bigger.  Bigger in everything.  Making it grow!'' --- Holly Purves

``I think it'd be good to get the young people involved in, er, whatever I'm standing for\ldots~General Council! Yeah!'' --- Naomi Wilkins

``I'm good at organising emails because I like males, and\ldots~Es.'' --- Toby Attrill

``Fist the bump.'' --- Sophie Holden

``Hey, Saskia, I've got a new business! [What is it?]  It's having open-space discussion at Althing! [How's it going for you?]  It's pretty a-musing!'' --- Jack Yeo

``Will a whole 5cm make that much difference?'' / ``Depends what it is you're talking about.'' --- Louise Delmege / Nick FitzGerald

``Delete the second Treasurer.'' --- Joe Bowler

``When I get home, I'm going to cry, drink wine, and bitch about the passing of time.'' --- Ruori McIntyre

``There's a bit of bitch for everyone.'' --- Nick FitzGerald

% Buzzword bingo
% Josh ``The role of the Chair'' 12.07
% Louise -- ``I worry'' 12.08, Sophie Holden 12.42
% Saskia/Naomi -- ``No job descriptions.''/``U wot m8.''
% Ryan -- ``I feel that'' 17.54, Tom Brooks 19.48
\end{document}
