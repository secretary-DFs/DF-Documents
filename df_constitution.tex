\documentclass[a4paper, 12pt]{report}

\usepackage{setspace}
\usepackage{enumerate}

\usepackage[scale=0.85]{geometry}

\usepackage{fontspec}
\setmainfont[Ligatures=TeX]{Linux Libertine}

\usepackage[british]{babel}

\newcommand{\HRule}{\rule{\linewidth}{0.5mm}}

\setcounter{tocdepth}{2}
\setcounter{secnumdepth}{5}

\renewcommand{\thechapter}{\Alph{chapter}}

\begin{document}

\begin{titlepage}
\begin{center}

\doublespacing
\vspace*{1.5cm}

\textsc{\textbf{ \LARGE Rules and Governing Documents for a Branch of the Woodcraft Folk, which is a Section }}

\vspace{1.5cm}

Rules as delegated by the General Council of the Woodcraft Folk incorporated as a Company Limited by Guarantee, Company number: 8133727

Charity number in England and Wales: 1148195 and in Scotland: SC039791

\vspace{1.5cm}

\HRule

\vspace{1.5cm}

\textsc{\textbf{\Large Rules of Procedure of the District Fellows Section of the Woodcraft Folk }}

Referred to as the ``DF Movement''

\vspace{1.5cm}

\HRule

\vspace{3cm}

Last amended and adopted by DF Althing 2012 (8th September 2012)

\vspace{0.5cm}

Approved by the Chairperson of General Council (signature) \underline{ \hspace{5cm} }

\end{center}
\end{titlepage}

\tableofcontents

\renewcommand{\chaptername}{Section}
\chapter{}
Section A can only be amended with the prior approval of General Council. It outlines the powers, provisions, membership and administrative affairs of the Section, which the Woodcraft Folk General Council has delegated.

\section{Provisions}
\label{sec:provisions}
\subsection{Interpretation and limitation}
\subsubsection{}
In these rules of procedure any defined terms used are set out in the Articles of the Woodcraft Folk: Article 49 (Defined terms). If any dispute arises in relation to the interpretation of these of these rules, the Standing Orders Committee shall resolve it.
\subsubsection{}
If there is any inconstancy between these Rules and:
\begin{enumerate}[\hspace{0.5cm}(a)]
\item regulations laid down by the General Council,
\item policies laid down by the Annual Gathering,
\item the Aims Principles and Programme of the Woodcraft Folk, or
\item the Articles of the Woodcraft Folk,
\end{enumerate}
the latter shall always take precedence over the former.

\subsubsection{}
For the avoidance of doubt, the DF Movement is a constituent part of the Woodcraft Folk and powers are delegated by the General Council under the Articles and may be revoked or amended at any time by a simple majority of the General Council under the Articles.

\subsection{Name and area covered}
The name of this section of the Woodcraft Folk is the District Fellows Movement. In these Articles it is called the ``DF Movement''. It shall cover all members of the Woodcraft Folk who fulfil the requirements under article \ref{sec:members} of these rules.

\subsection{Registered address}
The DF Movement shall register an address within the United Kingdom with the Woodcraft Folk's General Council for the purpose of communications.

\subsection{Objects}
\label{sec:objects}
The objects of the DF Movement shall be to further the objects of the Woodcraft Folk as outlined in the Articles of Association. Specifically the DF Movement shall fulfil the advancement of education and the empowerment of children and young people for the public benefit by:
\begin{enumerate}
\item co-ordinating and supervising all the activities of the Woodcraft Folk for the DF age range, as determined by the committee;
\item initiating joint activities between Woodcraft Folk Groups and work with other age groups, areas and regions as appropriate;
\item initiating and promote Woodcraft Folk leadership training;
\item ensuring that only suitable individual members of the Woodcraft folk are entrusted with responsibilities of Group leaders and other offices of the DF Movement;
\item ensuring that all activities are consistent with the Aims and Objectives of the Woodcraft Folk;
\item regularly reporting its activities and finances to the General Council;
\item ensuring the return of accounts to the General Council of all groups in its area;
\item ensuring that all over 16 year old members are registered with the General Council;
\item implementing any regulations of the General Council and policy of the Annual Gathering, in particular in regards to Child Protection and Safeguarding;
\item working both nationally and internationally with like-minded organisations committed
to socialist principles and linked to our common heritage.
\end{enumerate}

\subsection{Powers of the organisation}
To further its objects, the DF Movement may on behalf of General Council and according to the policies of the Annual Gathering and regulations of the General Council:
\begin{enumerate}
\item provide services and facilities for Members within the age group of operation;
\item establish, support, promote, coordinate and operate groups for Members in the age group of operation;
\item raise funds and invite and receive contributions from any person provided that the DF Movement shall not carry out any taxable trading activities in raising funds;
\item purchase, lease, hire or receive property of any kind including land, buildings and equipment and maintain and equip it for use within the guidelines set out by General Council;
\item sell, manage, lease, mortgage, exchange, dispose of or deal with all or any of its property within the guidelines set out by General Council;
\item to employ staff only with the express permission of General Council;
\item make grants or loans of money and give guarantees;
\item set aside funds for special purposes or as reserves against future expenditure;
\item co-operate with other charities and bodies and exchange information and advice with them;
\item support fundraising activities carried out by its Members for charitable and social causes, including the provision of administrative support, banking facilities and acting as a holding trustee of any funds raised;
\item establish, co-ordinate, promote and operate camps, seminars, events, festivals trips and other activities for the development of its members;
\item alone or with other organisations:
\begin{enumerate}
\item carry out campaigning activities;
\item seek to influence public opinion;
\item make representations to and seek to influence governmental bodies and other
bodies and institutions
\item develop, reform and implement appropriate policies, legislation and regulations, provided that all such activities shall be confined to the activities which a charity may properly undertake and provided that the organisation complies with any guidance published by the Charity Commission;
\end{enumerate}
\item open and operate banking accounts and other facilities for banking and draw,
accept, endorse, negotiate, discount, issue or execute negotiable instruments such
as promissory notes or bills of exchange according to the regulations set out by
General Council providing that:
\begin{enumerate}
\item funds raised shall only be paid into accounts which have been notified and
approved by the General Council; and
\item all expenditure shall be approved by either the General Council or two committee
members; and
\item shall be applied only in furthering the objects.
% TODO: This doesn't make grammatical sense.
\end{enumerate}
\end{enumerate}

\section{Members}
\label{sec:members}
\subsection{Becoming and Individual Member}
\label{sec:indivmembers}
Individual Members of the DF Movement shall be as follows:
\begin{enumerate}[\hspace{0.5cm}(a)]
% TODO: Make this between 16 and 20 inclusive
\item \label{item:normalmember} a person who is aged between 16 and 20 and who has paid the annual subscription laid down from time to time by the Annual Conference of the Woodcraft Folk and signed up to the Aims, Principles and Programme of the Woodcraft Folk; or
% TODO: Let Lloyd know this is unconstitutional as per WcF constitution
% TODO: Ask GC to remove Treasurer bit as it's unnecessary
\item  \label{item:greyarea} a person who has taken on specific responsibility for the DF Movement or been elected on a committee, and is aged between 21 and 22, or is the Treasurer of the DF Movement aged 23, and who has paid the annual subscription laid down from time to time by the Annual Conference of the Woodcraft Folk and signed up to the Aims, Principles and Programme of the Woodcraft Folk.
\end{enumerate}

% TODO: This is structurally bad, suggest the following:
%An Individual Member of the DF Movement shall be a person who has paid the annual subscription laid down from time to time by the Annual Conference of the Woodcraft Folk and signed up to the Aims, Principles and Programme of the Woodcraft Folk and:
%\begin{enumerate}
%\item is aged between 16 and 20; or
%\item  has taken on specific responsibility for the DF Movement or been elected on a committee, %and is aged between 21 and 22, or is the Treasurer of the DF Movement aged 23.
%\end{enumerate}

\subsection{Membership rights}
\subsubsection{}
Members of the DF Movement organisation shall be entitled to the benefits set out in the Aims, Principles and Programme.
\subsubsection{}
Only members outlined in \ref{sec:indivmembers} shall have the right to vote on each motion and
candidates for election.
\subsubsection{}
All members of the DF Movement must support the Aims, Principles and Programme of the Woodcraft Folk.
\subsubsection{}
All members, that are required to do so, shall have completed a full child protection and safeguarding produces as laid out by General Council's regulations.

\subsection{Termination of Membership}
Membership shall not be transferable and shall cease on death. A Member shall cease to be
a Member of the organisation if:
\begin{enumerate}[\hspace{0.5cm}(a)]
\item they cease to be 16 to 20 in clause \ref{sec:indivmembers}(\ref{item:normalmember}) or;
\item they cease to be 21 to 22 in clause \ref{sec:indivmembers}(\ref{item:greyarea}) or;
\item they cease to hold a position of responsibility in clause \ref{sec:indivmembers}(\ref{item:greyarea}); or
\item they resign their membership in writing; or
\item their membership is terminated according to the Woodcraft Folk's Article 2.6 and associated regulations set out by the General Council.
\end{enumerate}

\section{Administrative Arrangements and Other Clauses}
\label{sec:admin}
\subsection{Conflict of Interest}
\subsubsection{}
% TODO: Ask GC to clarify "Committee" throughout Section A
Whenever a matter is to be discussed at a meeting or decided and a Committee or Regional Council member has a Conflict of Interest in respect of that matter then they must:
\begin{enumerate}[\hspace{0.5cm}(a)]
\item withdraw from the debate and only remain for such part of the meeting as in the view of the other committee members is necessary to inform the debate;
\item not be counted in the quorum for that part of the meeting or decision-making process; and
\item withdraw during the vote and have no vote on the matter.
\end{enumerate}
\subsubsection{}
All member of the DF Committee and DF Regional Council shall abide by the Conflict of Interest Policy as laid down from time to time by the General Council.
\subsubsection{}
No member of the DF Committee or DF Regional Council may be employed or otherwise personally benefit from any transaction of the Woodcraft Folk without prior approval of the General Council.

\subsection{Minutes}
\subsubsection{}
The Secretary shall keep minutes of all:
\begin{enumerate}[\hspace{0.5cm}(a)]
\item appointments of officers made by the DF Movement;
\item resolutions of the DF Movement and of the committee; and
\item proceedings at meetings of the organisation and of the committee, including the names of the members present at each such meeting.
\end{enumerate}
\subsubsection{}
The minutes of Committee and Regional Council meetings must be kept for at least ten years from the date of the meeting, resolution or decision.
\subsubsection{}
The minutes of the meetings shall normally be considered open and shall be available to the Members on the organisation's website, except where those minutes relate to any reserved or confidential matters, including without limitation staff-related or disciplinary matters. The General Council may request all minutes, including those regarding confidential matters.

\subsection{Finances}
It is DF Committee as a whole that is responsible for the finances of the DF Movement with
the day-to-day operation in the hand of the Treasurer and the other authorised signatories.
\subsubsection{}
The Treasurer shall maintain accounting records and report on finances to each Committee meeting and quarterly to General Council.
\subsubsection{}
All expenditure shall be made only against agreed budgets set annually or for each event/activity.
\subsubsection{}
Procedures required by the General Council shall be followed including ensuring that all payments are signed by two signatories who are not related parties.
\subsubsection{}
Expenditure using restricted funds received will be accounted for clearly separate from that of other payments.
\subsubsection{}
The Treasurer shall be responsible for preparing annual accounts (Income and Expenditure Account and Balance Sheet) on a calendar year basis and submitting them within the time required by the General Council for consolidation as part of the charity as a whole.
\subsubsection{}
The Treasurer shall provide the person appointed to inspect or audit the accounts with the accounts, working papers, and all assistance that may be needed.

\subsection{Dissolution}
\subsubsection{}
If the committee decides that it is necessary or advisable that the DF Movement shall be dissolved, it shall call a General Meeting of the DF Movement by giving 7 days' notice in writing to each member and the General Council stating the terms of any resolution to be proposed.
\subsubsection{}
If it is decided at the Althing by a simple majority of those present and voting that the DF Movement shall be dissolved, the Committee shall wind up the DF Movement's affairs informing General Council of the resolution.
\subsubsection{}
Any assets remaining after the satisfaction of any proper debts and liabilities shall be given or transferred to The Woodcraft Folk, for development of work in the area in which the funds were raised. In the case of any restrictions of monies or assets held by the DF Movement the General Council shall determine who any funds should be distributed to in the fulfilment of the respective restrictions.
\subsubsection{}
A copy of the statement of accounts, or account and statement, for the final accounting period of the DF Movement should be sent to the Woodcraft Folk's registered office.

\subsection{Amending the Rules of Procedure}
\subsubsection{}
An Althing may amend the provisions of this deed, provided that:
\begin{enumerate}[\hspace{0.5cm}(a)]
\item no amendment may be made which limits of alters any powers or provisions relating to the General Council or Annual Conference of the Woodcraft Folk; and
\item no amendment may be made to clause \ref{sec:provisions} (Provisions), clause \ref{sec:members} (Members), or this clause \ref{sec:admin} (Administrative arrangements and other clauses) without the prior consent in writing of the General Council of the Woodcraft Folk; and
\item no amendment may be made whose effect breaks the Aims, Principles and Programme of the Woodcraft Folk, Policies of the Annual Conference or regulations of the General Council.
\end{enumerate}
\subsubsection{}
The committee must send to the General Council a certified copy of the Rules of Procedure noting any amendment made under this clause within two months of it being made for ratification by the General Council or their delegated officer.

\chapter{}
\section{DF Committee}
\subsection{}
There shall be a DF Committee (`Committee') composed of DFs and elected by DFs to oversee the implementation of the Movement's objectives as set out in article \ref{sec:objects}.

\subsection{Powers}
\subsubsection{}
To be the decision-making body for all DFs of the Woodcraft Folk subject to the rules and regulations of the Woodcraft Folk.
\subsubsection{}
To secure and administer funding for the work of the DF Movement and to invite and receive contributions provided that in raising funds Committee do not undertake any substantial permanent trading activities and conform to any relevant requirements of the law.
\subsubsection{}
To co-operate with other charities, voluntary bodies and statutory authorities operating in furtherance of the objects or of similar charitable purposes and to exchange information and advice with them.
\subsubsection{}
In order to achieve the above purposes, the DF business events (see section \ref{sec:business}) may from time to time establish sub-committees and working groups with delegated powers.

\subsection{Committee Roles and Officers}
\subsubsection{}
\label{sec:cttemembership}
Committee shall consist of at most thirteen or fourteen members, with the number varying according to the rules set out in \ref{sec:treasurerrole} and \ref{sec:moreroles}. It shall have the following composition:

A DF representative for each of the following positions:
\begin{itemize}
\item Affiliations Representative
\item Campaigns Representative
\item Communications Representative
\item Events Representative
\item Fundraising Representative
\item International Opportunities Office
\item Projects Representative
\item Training Representative
\item Treasurer
\item Secretary of the Movement
\item Shadow Treasurer (If this is applicable according to the rules set out in \ref{sec:treasurerrole})
\item Chair of Regional Council
\item Lay Member 1
\item Lay Member 2
\end{itemize}

Additionally, the representatives for any of these positions may take on a secondary role as outlined in section \ref{sec:secondaryroles}.

\subsection{Election of Committee Members}
\label{sec:election}
\subsubsection{}
Elections for Committee (with the exception of Chair of Regional Council) shall be held at Althing (see section \ref{sec:althing}), and nominations be may made up until the point at which hustings for the positions begins.

\subsubsection{}
The Chair of Regional Council will be elected for a term of one year from and by Regional Council by the method set out in \ref{sec:electionprocedure} at Old/New (see paragraph \ref{sec:rcchair}).

\subsubsection{}
All candidates nominated for positions on Committee must be members of the Woodcraft Folk and DFs, as set out above.

\subsubsection{}
All Committee members, apart from Treasurer and Shadow Treasurer (as set out in \ref{sec:treasurerrole} are elected to serve for a term of two years, from the Old/New immediately before which they were elected to the Old/New held two years later.

\subsubsection{}
\label{sec:electionprocedure}
The election system to be used at Althing shall be as follows:
\begin{itemize}
\item All votes shall rank all the candidates in order of preference.
\item The number of votes for each candidate shall be the number of times that candidate was placed first.
\item The candidate with the least votes will be eliminated and their votes redistributed until there is one candidate remaining.
\item This process will then be repeated with the remaining candidates until it is known which candidate came in each position. The first n where n is the number of available positions will be elected.
\end{itemize}

\subsubsection{}
\label{sec:moreroles}
The same person can fill more than one role.

\subsubsection{}
If a position on Committee remains or becomes vacant, Committee may co-opt at their discretion a member of the DF Movement to fill the position.

\subsection{Old/New and Secondary Roles}
\label{sec:secondaryroles}
\subsubsection{Old/New}
\label{sec:oldnew}
\paragraph{} Annually, not more than two months after Althing, DF Committee and Regional Council shall hold a closed meeting called `Old/New'.
\paragraph{} At this meeting DF Committee shall elect, from amongst their number, individuals to hold the following Secondary Roles in addition to the role they were elected to at Althing:
\begin{itemize}
\item Staff Liaison
\item Sustainability Representative
\item Vice Chair of DF Committee
\item Chair of DF Committee
\end{itemize}

\paragraph{} The term of a Secondary Role shall be one year.

\paragraph{} Additionally, two members shall be elected from DF Committee and Regional Council combinded to represent the Movement on the Woodcraft Folk's General Council for one year.

\subsection{Joining and Leaving Committee}
\subsubsection{Vacation of a Role}
A person may no longer serve on Commitee if:
\begin{enumerate}[\hspace{0.5cm}(a)]
\item they cease to be a member of the organisation; or
\item they stand down by notice to the DF Movement; or
\item they lose a vote of no confidence as set out in \ref{sec:noconfidence}
\end{enumerate}
\subsubsection{Vote of No Confidence}
\label{sec:noconfidence}
\paragraph{} Any DF may call a vote of no confidence on a member of Committee; this vote will be taken by Committee (excluding the member in question) and will require an 2/3rds majority vote to pass.
\paragraph{} A vote may be helt at any time as long as the Committee member in question was given at least 10 days' notice of the vote, and is offered to present their case to Committee either in person or in writing before the vote. Every member of Committee must be involved in the discussion and the vote.
\paragraph{} In the case of physical or mental incapacity, lack of attendance, or lack of report production to Committee meetings twice in a row, a vote of no confidence shall take place automatically.
\subsubsection{Replacement of a Role}
\label{sec:rolereplacement}
\paragraph{} If any vacancies arise on Committee, Committee may co-opt any member of the Movement to fill the vacancy. This will require a 2/3rds majority decision to pass.
\paragraph{} If a person holding a 2-year role, with the exception of the Campaigns Rep and either Lay Member, stands down mid-term at an Althing, an election shall be held in the usual way for the role, and the role will be held by the winner of the election for a further 2 years.
\paragraph{} Since campaigns are elected for 2 years at a time, if a new Campaigns Representative is elected mid-term, they will only serve up until the end of the campaign underway when they were elected, i.e. for 1 year.
\paragraph{} This does not apply to Lay Members, as there should always be one Lay Member elected at each Althing. Therefore if a Lay Member stands down mid-term, the person elected to replace them will only serve until the end of the term of the person they are replacing, i.e. for 1 year.

\subsection{Role of the Treasurer}
\label{sec:treasurerrole}
\subsubsection{}
The Shadow Treasurer and Treasurer role are both uniquely different to other Committee roles. It has been recognised that the role of Shadow Treasurer and Treasurer are particularly challenging and important to the movement. This system set out below has therefore been adopted and supersedes the provisions set out above.
\paragraph{}
The Shadow Treasurer is elected for a term of one year from the floor of Althing. As with other roles, the incumbent Shadow Treasurer may re-stand.
\paragraph{}
The Treasurer is elected for a term of one year by the floor of Althing. Only previous Shadow Treasurers (or the incumbent or previous Treasurers) may stand.
\paragraph{}
This will mean that any Treasurer elected at an Althing will have already been a Shadow Treasurer at some point.
\paragraph{}
The Shadow Treasurer will work with the Treasurer over the next year to learn the skills necessary to carry out the role effectively.
\paragraph{}
Any candidate for Shadow Treasurer must be under 20 years old at Althing, so that after completing a term as Shadow Treasurer, they may still stand for the position of Treasurer.
\paragraph{}
The Treasurer shall produce quarterly financial reports of the DF accounts to be sent to both DF Committee and General Council.
\paragraph{}
\label{sec:treasurerreport}
Before the election of a new Treasurer and Shadow Treasurer, Committee shall submit a report to Althing on their success in the role which shall be included in the minutes. This is to ensure that if that person stands for the role of Treasurer in future years, an informed choice may be made by the floor of Althing.
\subsubsection{No Confidence in the Treasurer}
\label{sec:treasurernoconfidence}
\paragraph{} If the Treasurer position becomes vacant, by a vote of no confidence or otherwise, the Shadow Treasurer shall become the Treasurer and a new Shadow Treasurer shall be co-opted by DF Committee in the usual way (\ref{sec:rolereplacement}).
\paragraph{} If the Shadow Treasurer is unable to fully take on the role of Treasurer, the General Secretary, Chair and Treasurer of the Woodcraft Folk will take on the Treasurer's responsibilities until the Shadow Treasurer has had sufficient training.
\paragraph{} The decision to either hand over or take back power of the accounts will be made by DF Committee.

\section{Regional Council}
\subsection{Powers}
\subsubsection{}
There shall be a Regional Council composed of DFs and elected by DFs to oversee the development of regions as defined in section \ref{sec:rcmembership}.
\subsubsection{Regional Council shall have the following powers:}
\paragraph{}
To develop their region with a view to increasing membership and participation.
\paragraph{}
To act as a point of contact between members of their region or nation and Committee, through the Chair of Regional Council.
\paragraph{}
To provide a smooth transition between ``Venturers'' (13-15 years old inclusive members of the Woodcraft Folk) and DF groups in their region.
\paragraph{}
To assist in the development of new DF groups in their region.
\paragraph{}
To ensure representation of their region at ``Annual Gathering'', the Woodcraft Folk's annual general meeting.

\subsection{Membership of Regional Council}
\label{sec:rcmembership}
\subsubsection{Regions and Nations}
\label{sec:regionsandnations}
\paragraph{} Regional Council shall consist of at most ten or eleven members. It shall have the following composition: A DF representative from each of the following regions or nations:
\begin{enumerate}[\hspace{0.5cm}(a)]
\item Scotland
\item North West England
\item North East England
\item Wales
\item Midlands
\item Eastern England
\item South West Area Network (SWAN)
\item Oxford, Swindon and Gloucester (OSG) 
\item South East England
\item London
\end{enumerate}
\paragraph{} If two candidates cannot be found to stand for SWAN and OSG between them, one candidate may stand for both positions.
\subsubsection{Chair}
\label{sec:rcchair}
\paragraph{} At Old/New, Regional Council shall elect from amongst their number, or from the previous years' Regional Council, a Chair.

\paragraph{} The Chair of Regional Council shall become a member of DF Committee for a term of one year.

\subsection{Election of Regional Council members}
\subsubsection{The following rules shall apply to the election of DFs to Regional Council:}
\paragraph{}
\label{sec:regionalresidency}
Members of Regional Council must expect the be resident within the region to which they seek election for at least 6 months of the following year.
\paragraph{}
Elections for Regional Council shall be held at Althing, and nominations may be made up until the point at which a husting for the position begins.
\paragraph{}
All candidates nominated for positions on Regional Council must be members of the Woodcraft Folk and the DF Movement, as set out in \ref{sec:members}.
\paragraph{}
Regional Council members are elected to serve for a term of one year, from the Old/New immediately before which they were elected, to the Old/New held one year later.
\paragraph{}
If any Regional Council posts are unfilled then nominations shall be open to those not meeting the requirements as set out in section \ref{sec:regionalresidency}. If the post remains unfilled, Committee can co-opt a DF to fill this role.
\paragraph{}
The election system to be used is that set out in \ref{sec:electionprocedure}.

\subsection{Annual Gathering}
At the Woodcraft Folk's Annual Gathering, there shall be one vote allocated to a representative for each of the Woodcraft Folk's regions, distributed among Regional Council as follows:
\begin{enumerate}[\hspace{0.5cm}(a)]
\item Scotland: Scotland Representative
\item Northern: to be shared between North East and North West England Representatives
\item Wales: Wales Representative
\item Midlands: Midlands Representative
\item Eastern: Eastern England Representative
\item South West: SWAN and OSG Representatives
\item South East: South East Representative
\item London: London Representative
\end{enumerate}
\section{Business Events}
\label{sec:business}
\subsection{Althing (Annual General Meeting)}
\label{sec:althing}
\subsubsection{}
Each year, there shall be an Althing, open to all DFs, which will be the Movement's Annual General Meeting.
\subsubsection{}
In choosing the date for Althing, Committee shall ensure that there is sufficient time to allow the execution of motions that call for the submission of motions to the Woodcraft Folk's Annual Gathering.
\subsubsection{}
All DFs attending Althing shall have one vote on each motion and may vote in elections. All members of Committee or Regional Council who are in the `Grey Area' (clause \ref{sec:indivmembers}(\ref{item:greyarea})) also have one vote on each motion and may vote in elections. No other attendees of Althing may vote on motions or in elections.
\subsubsection{General Council Representatives}
Althing shall elect 2 representatives to serve on the Woodcraft Folk's General Council for a term of one year. The election procedure shall be the same as is used for election of Committee and Regional Council members (\ref{sec:electionprocedure}).
\subsubsection{The general business of Althing shall be as follows:}
\paragraph{}
Ratification of minutes of the previous Althing.
\paragraph{}
Reports from Committee members and members of Regional Council describing their activities over the past year.
\paragraph{}
A financial report.
\paragraph{}
Discussion of and voting on motions.
\paragraph{}
Discussion of ``musings'', which is an open space discussion where members may bring up issues and actions which may be determined form the outcome of the debate.
\paragraph{}
The Treasurer and Shadow Treasurer report from Committee, as detailed in \ref{sec:treasurerreport}
\paragraph{}
Nominations, hustings and elections for places on Committee and Regional Council.
\paragraph{}
Any other business.
\paragraph{}
Workshops, discussions and other sessions of interest to members.
\subsubsection{}
Motions to Althing may be submitted before or during Althing and amendments to motions may be submitted at any time before the motion in voted upon.

\subsection{Things (General Meetings in a particular Region)}
\subsubsection{}
Each year, there shall be three DF business events called `Things', the dates and venues for which shall be announced by Committee.
\subsubsection{}
The venue of business meetings shall move between the regions which are represented on Regional Council as defined in \ref{sec:regionsandnations}, apart from SWAN and OSG, which will have one Thing for the whole South West region.
\subsubsection{The general business of Things shall be as follows:}
\paragraph{}
Reports from Committee members describing their activities since the last business event, be it a Thing or Althing.
\paragraph{}
A financial report.
\paragraph{}
Discussion of forthcoming events, campaigns, etc. 
\paragraph{}
Any other business.
\paragraph{}
Discussions and workshops about issues that are of interest to DFs. 

\subsection{Quorum}
No business shall be transacted at any Althing or Thing unless a quorum of voting
members are present.
\subsubsection{Althing} The quorum for decisions taken at Althing shall be thirty-three registered DFs, or twice the number of existant Committee members plus one, whichever is the lesser.
\subsubsection{Things} The quorum for decisions taken at a Thing shall be half the number of people present at the Thing plus one. Quorum is required for all non-closed Committee discussions at Things.

\end{document}
