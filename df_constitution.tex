\documentclass[a4paper, 12pt]{article}

\usepackage{setspace}

\setlength{\oddsidemargin}{0in} \setlength{\evensidemargin}{0in}
\setlength{\textwidth}{6.2in}
\setlength{\topmargin}{-1.0in} \setlength{\textheight}{10.2in}

\renewcommand{\familydefault}{\sfdefault}

\title{DF Constitution}
\author{The DF Movement}
\date{2nd May 2012}

\setcounter{tocdepth}{2}
\setcounter{secnumdepth}{5}

\begin{document}

\maketitle
\tableofcontents

\section{Name}
\subsection{}
The name of the Association shall be District Fellows (`DFs').

\section{Object}
\subsection{}
The object of the Association shall be to support the work of young members of those Woodcraft Folk aged between sixteen and twenty years inclusive, so as to develop their physical and mental potential so they may grow to full maturity both as individuals and members of society and that everyone’s conditions in life may be improved based not on wealth but a common unity as humankind.
\subsection{}
Furthermore, the District Fellows shall support the work of the Woodcraft Folk as an educational movement for young people, to develop their self-confidence and activity in society with the aim of building a world built on equality, friendship, peace, co-operation and environmental sustainability. 
\subsection{}
This educational objective shall not be limited to members of District Fellows, but will also be aimed at young people in general, both within and without the Woodcraft Folk.
\subsection{}
The association shall work not only in improving the lives of its membership but campaigning and educating for a more social world. It will work both nationally and internationally with like-minded organisations committed to socialist principles and linked to our common heritage.

\section{Membership}
\subsection{}
\label{sec:membership}
Individuals -- Membership of the Woodcraft Folk is open to any person interested in furthering the objects and who has paid the annual subscription laid down from time to time by the Annual Conference of the Woodcraft Folk, and all Woodcraft Folk members aged 16 to 20 years inclusive shall be regarded as District Fellows.
\subsection{The Grey Area}
\label{sec:greyarea}
\subsubsection{}
When a DF has taken on a specific and significant responsibility for the movement, they may remain a DF past their 21st birthday. This includes, for example, organising an event, being a member of DF Committee, or being a member of the Regional Development Committee.
\subsubsection{}
With the exception of the Treasurer (as outlined in \ref{sec:treasurergreyarea}), if a member of DF Committee or the Regional Development Committee is aged 21 at an Althing, they must stand down.

\section{DF Committee}
\subsection{Powers}
\subsubsection{}
There shall be a DF Committee (`Committee') composed of DFs and elected by DFs to oversee the implementation of the Association’s objectives as set out above.

\subsubsection{Committee shall have the following powers:}
\paragraph{}
To be the decision-making body for all District Fellows of the Woodcraft Folk subject to the rules and regulation of the Woodcraft Folk.
\paragraph{}
To secure and administer funding for the work of the District Fellows and to invite and receive contributions provided that in raising funds Committee do not undertake any substantial permanent trading activities and conform to any relevant requirements of the law.
\paragraph{}
In order to achieve the above purposes, the District Fellow business events (see below) may from time to time establish sub-committees and working groups with delegated powers.
\paragraph{}
To co-operate with other charities, voluntary bodies and statutory authorities operating in furtherance of the objects or of similar charitable purposes and to exchange information and advice with them.

\subsection{Membership of Committee}
\subsubsection{}
\label{sec:cttemembership}
Committee shall consist of at most thirteen or fourteen members, with the number varying according to the rules set out in \ref{sec:moreroles} and \ref{sec:treasurerrole}. It shall have the following composition:

A DF representative for each of the following positions:
\begin{itemize}
\item Affiliations Representative
\item Campaigns Representative
\item Communications Representative
\item Fundraising Representative
\item Projects Representative
\item Shadow Treasurer (If this is applicable according to the rules set out in \ref{sec:treasurerrole})
\item Regions and Nations Representative
\end{itemize}
A further 6 people will be elected as committee members.

\subsubsection{}
Any member of DFs may call a vote of no confidence on a committee member; this vote will be taken by committee (excluding the member in question) and will require an unanimous decision to pass.

\subsection{Election of Committee Members}
\label{sec:election}
\subsubsection{The following rules shall apply to the election of DFs to Committee:}
\paragraph{}
Elections for Committee shall be held at the annual Althing (see below), and nominations be may made up until the point at which hustings for the positions begins.

\paragraph{}
The Regions and nations representative will be elected for a term of one year from and by the regional development committee by the method set out in \ref{sec:electionprocedure} at Althing or through a teleconference to be held within two weeks of Althing.

\paragraph{}
All candidates nominated for positions on Committee must be members of the Woodcraft Folk and District Fellows, as set out above.

\paragraph{}
Committee members are elected to serve for a term of two years, from the Althing at which they are elected to the Althing held two years later.

\paragraph{}
Committee members may continue to serve on Committee if their term includes their 21st birthday,  however they must stand down at the next Althing after their 21st birthday.  They are also therefore entitled to attend those District Fellows events which would not ordinarily be open to them if they were not members of Committee.

\paragraph{}
\label{sec:electionprocedure}
The election system to be used at Althing shall be as follows:
\begin{itemize}
\item All votes shall rank all the candidates in order of preference.
\item The number of votes for each candidate shall be the number of times that candidate was placed first.
\item The candidate with the least votes will be eliminated and their votes redistributed until there is one candidate remaining.
\item This process will then be repeated with the remaining candidates until it is known which candidate came in each position. The first n where n is the number of available positions will be elected.
\end{itemize}

\paragraph{}
\label{sec:moreroles}
The same person can fill more than one role.

\subsection{Roles to be undertaken by Committee members}
\subsubsection{}
Annually, not more than two months after the Althing, Committee shall hold a closed meeting where its members shall elect from amongst their number, individuals to be responsible for particular areas.
\subsubsection{}
The following roles shall be invariable and be filled by a Committee member every year: Chairperson, Treasurer and Secretary.

\subsection{Role of the Treasurer}
\label{sec:treasurerrole}
\subsubsection{}
It has been recognised that the role of the Treasurer is particularly challenging. The system set out below has therefore been adopted, and supersedes the provisions set out above:
\paragraph{}
In the Treasurer's second year in the role an election will be held for Shadow Treasurer, this election shall be from the floor of Althing. This election shall be held as outlined in \ref{sec:election}.
\paragraph{}
The Shadow Treasurer will work together with the Treasurer over the year until the next Althing to learn the skills necessary to carry out the role effectively. The Shadow Treasurer will then become Treasurer and serve for a further two Althings unless the current Treasurer contests this as set out in.
\paragraph{}
The outgoing Treasurer may stand in place of the Shadow Treasurer. If the current Treasurer is elected than their role will be extended for one year and an election will be held at the current Althing for a Shadow Treasurer. The outgoing Shadow Treasurer may re-stand for Shadow Treasurer.
\paragraph{}
This will mean that that any candidate for Treasurer will already have been Shadow Treasurer and will therefore serve the equivalent of one-and-a-half terms on Committee, meaning that every other year there will be an `extra' member on Committee.
\paragraph{}
\label{sec:treasurergreyarea}
If the Treasurer is 21 at an Althing, they are not mandated to stand down as with other committee roles, however there must be an election for the role of Shadow Treasurer unless the role is already filled.

\section{Regional Development Committee}
\subsection{Powers}
\subsubsection{}
There shall be a Regional Development Committee composed of DFs and elected by DFs to oversee the development of regions as defined in section \ref{sec:rdcmembership}.
\subsubsection{Regional Development Committee shall have the following powers:}
\paragraph{}
To develop their region with a view to increasing membership and participation.
\paragraph{}
To act as a point of contact between members of their region and the regions and nations representative on (DF) committee.
\paragraph{}
To provide a smooth transition between ``Venturer'' (13-15 years old inclusive members of the Woodcraft Folk) and DF groups in their region.
\paragraph{}
To assist in the development of new DF groups in their region.
\paragraph{}
To ensure representation of their region at ``Annual Gathering'' the Woodcraft Folks annual general meeting.

\subsection{Membership of the Regional Development Committee}
\label{sec:rdcmembership}
\subsubsection{}
Regional Development Committee shall consist of nine members. It shall have the following composition: A DF representative from each of the following regions or nations: Scotland, North West England, North East England, Wales, Midlands England, Eastern England, South West England, London and South East England.
\subsection{Election of Regional Development Committee members}
\subsubsection{The following rules shall apply to the election of DFs to the Regional Development Committee:}
\paragraph{}
\label{sec:regionalresidency}
Members of the Regional Development Committee must expect the be resident within the region to which they seek election for at least 6 months of the following year.
\paragraph{}
Elections for the Regional Development Committee shall be held at Althing, and nominations may be made up until the point at which a husting for the position begins.
\paragraph{}
All candidates nominated for positions on the Regional Development Committee must be members of the Woodcraft Folk and DFs, as set out in \ref{sec:membership}.
\paragraph{}
Regional Development Committee members are elected to serve for a term of one year, from the Althing at which they are elected, to the Althing held one year later.
\paragraph{}
Members of the Regional Development Committee may continue to serve on the Regional Development Committee if their term includes their 21st birthday, however they must stand down at the next Althing after their 21st birthday. Members of Regional Development Committee shall be entitled to attend business events. This in no way alters their membership status.
\paragraph{}
If any Regional Development Committee posts are unfilled then nominations shall be open to those not meeting the requirements as set out in section \ref{sec:regionalresidency}. If the post remains unfilled committee can co-opt a DF to fill this role.
\paragraph{}
The election system to be used is that set out in \ref{sec:electionprocedure}.

\section{Meetings}
\subsection{The Althing}
\subsubsection{}
Each year, there shall be an Althing, open to all DFs, which will be the  Association’s Annual General Meeting.
\subsubsection{}
In choosing the date for Althing, Committee shall ensure that there is sufficient time to allow the execution of motions that call for the submission of motions to the Woodcraft Folk’s Annual Gathering.
\subsubsection{}
All District Fellows attending the Althing shall have one vote on each motion.  This shall include members of Committee and Regional Development Committee who are aged twenty-one or over, but not any others invited to the Althing by Committee.
\subsubsection{The general business of the Althing shall be as follows:}
\paragraph{}
Reports from Committee members and members of regional development committee describing their activities over the past year.
\paragraph{}
A financial report.
\paragraph{}
Discussion of motions.
\paragraph{}
Nominations, hustings and elections for places on Committee.
\paragraph{}
Any other business.
\paragraph{}
Workshops, discussions and other sessions of interest to members.
\subsubsection{}
Motions to Althing may be submitted before or during Althing and amendments to motions may be submitted at any time before the motion in voted upon.
\subsubsection{}
The quorum for decisions taken at the Althing shall be thirty-three registered DFs, or twice the number of Committee members plus one, whichever is the lesser.

\subsection{Business Meetings}
\subsubsection{}
Each year, there shall be three DF Forums, the dates and venues for which shall be announced by Committee. 
\subsubsection{}
The venue of business meetings shall move between the regions which are represented on Committee, as defined in \ref{sec:cttemembership}.
\subsubsection{}
At the Winter Wonderland event, nominations, hustings and elections for two DF representatives to the Woodcraft Folk’s General Council shall take place.
\subsubsection{The general business of the Forum shall be as follows:}
\paragraph{}
Reports from Committee members describing their activities since the last business event, be it the business meeting or the Althing.
\paragraph{}
A financial report.
\paragraph{}
Discussion of forthcoming events, campaigns, etc. 
\paragraph{}
Any other business.
\paragraph{}
Discussions and workshops about issues that are of interest to DFs. 

\end{document}
