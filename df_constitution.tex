\documentclass[a4paper, 11pt]{report}

\usepackage{dfdocs}

\newcommand{\HRule}{\rule{\linewidth}{0.5mm}}

\setcounter{tocdepth}{2}
\setcounter{secnumdepth}{5}

\renewcommand{\thechapter}{\Alph{chapter}}

% Use these for outputting to plain text (via catdvi)
% Use `catdvi -e 1 -U df_constitution.dvi | sed -re "s/\[U\+2022\]/*/g" | sed -re "s/([^^[:space:]])\s+/\1 /g" | tr -d "\f"`
%\usepackage[none]{hyphenat}
%\pagenumbering{gobble}
%\raggedright

\lhead{DF Constitution}

\begin{document}

\begin{titlepage}
\begin{center}

\doublespacing
\vspace*{1cm}

\textsc{\textbf{ \LARGE Rules and Governing Documents for a Branch of the Woodcraft Folk, which is a Section }}

\vspace{1cm}

Rules as delegated by the General Council of the Woodcraft Folk incorporated as a Company Limited by Guarantee, Company number: 8133727

Charity number in England and Wales: 1148195 and in Scotland: SC039791

\vspace{1cm}

\HRule

\vspace{1.5cm}

\textsc{\textbf{\Large Rules of Procedure of the District Fellows Section of the Woodcraft Folk }}

Referred to as the ``DF Movement''

\vspace{1.5cm}

\HRule

\vspace{2cm}

Last amended and adopted by DF Althing 2017 (9th August 2017)

\vspace{0.5cm}

Approved by the Chairperson of General Council (signature) \underline{ \hspace{5cm} }

\end{center}
\end{titlepage}

\tableofcontents

\renewcommand{\chaptername}{Section}
\chapter{}
Section A can only be amended with the prior approval of General Council. It outlines the powers, provisions, membership and administrative affairs of the Section, which the Woodcraft Folk General Council has delegated.

\section{Provisions}
\label{sec:provisions}
\subsection{Interpretation and limitation}
\subsubsection{}
In these rules of procedure any defined terms used are set out in the Articles of the Woodcraft Folk: Article 49 (Defined terms). If any dispute arises in relation to the interpretation of these of these rules, the Standing Orders Committee shall resolve it.
\subsubsection{}
If there is any inconstancy between these Rules and:
\begin{enumerate}[\hspace{0.5cm}(a)]
\item regulations laid down by the General Council,
\item policies laid down by the Annual Gathering,
\item the Aims Principles and Programme of the Woodcraft Folk, or
\item the Articles of the Woodcraft Folk,
\end{enumerate}
the latter shall always take precedence over the former.

\subsubsection{}
For the avoidance of doubt, the DF Movement is a constituent part of the Woodcraft Folk and powers are delegated by the General Council under the Articles and may be revoked or amended at any time by a simple majority of the General Council under the Articles.

\subsection{Name and area covered}
The name of this section of the Woodcraft Folk is the District Fellows Movement. In these Articles it is called the ``DF Movement''. It shall cover all members of the Woodcraft Folk who fulfil the requirements under article \ref{sec:members} of these rules.

\subsection{Registered address}
The DF Movement shall register an address within the United Kingdom with the Woodcraft Folk's General Council for the purpose of communications.

\subsection{Objects}
\label{sec:objects}
The objects of the DF Movement shall be to further the objects of the Woodcraft Folk as outlined in the Articles of Association. Specifically the DF Movement shall fulfil the advancement of education and the empowerment of children and young people for the public benefit by:
\begin{enumerate}
\item co-ordinating and supervising all the activities of the Woodcraft Folk for the DF age range, as determined by the committee;
\item initiating joint activities between Woodcraft Folk Groups and work with other age groups, areas and regions as appropriate;
\item initiating and promote Woodcraft Folk leadership training;
\item ensuring that only suitable individual members of the Woodcraft folk are entrusted with responsibilities of Group leaders and other offices of the DF Movement;
\item ensuring that all activities are consistent with the Aims and Objectives of the Woodcraft Folk;
\item regularly reporting its activities and finances to the General Council;
\item ensuring the return of accounts to the General Council of all groups in its area;
\item ensuring that all over 16 year old members are registered with the General Council;
\item implementing any regulations of the General Council and policy of the Annual Gathering, in particular in regards to Child Protection and Safeguarding;
\item working both nationally and internationally with like-minded organisations committed
to socialist principles and linked to our common heritage.
\end{enumerate}

\subsection{Powers of the organisation}
To further its objects, the DF Movement may on behalf of General Council and according to the policies of the Annual Gathering and regulations of the General Council:
\begin{enumerate}
\item provide services and facilities for Members within the age group of operation;
\item establish, support, promote, coordinate and operate groups for Members in the age group of operation;
\item raise funds and invite and receive contributions from any person provided that the DF Movement shall not carry out any taxable trading activities in raising funds;
\item purchase, lease, hire or receive property of any kind including land, buildings and equipment and maintain and equip it for use within the guidelines set out by General Council;
\item sell, manage, lease, mortgage, exchange, dispose of or deal with all or any of its property within the guidelines set out by General Council;
\item to employ staff only with the express permission of General Council;
\item make grants or loans of money and give guarantees;
\item set aside funds for special purposes or as reserves against future expenditure;
\item co-operate with other charities and bodies and exchange information and advice with them;
\item support fundraising activities carried out by its Members for charitable and social causes, including the provision of administrative support, banking facilities and acting as a holding trustee of any funds raised;
\item establish, co-ordinate, promote and operate camps, seminars, events, festivals trips and other activities for the development of its members;
\item alone or with other organisations:
\begin{enumerate}
\item carry out campaigning activities;
\item seek to influence public opinion;
\item make representations to and seek to influence governmental bodies and other
bodies and institutions
\item develop, reform and implement appropriate policies, legislation and regulations, provided that all such activities shall be confined to the activities which a charity may properly undertake and provided that the organisation complies with any guidance published by the Charity Commission;
\end{enumerate}
\item open and operate banking accounts and other facilities for banking and draw,
accept, endorse, negotiate, discount, issue or execute negotiable instruments such
as promissory notes or bills of exchange according to the regulations set out by
General Council providing that:
\begin{enumerate}
\item funds raised shall only be paid into accounts which have been notified and
approved by the General Council; and
\item all expenditure shall be approved by either the General Council or two committee
members; and
\item shall be applied only in furthering the objects.
% TODO: This doesn't make grammatical sense.
\end{enumerate}
\end{enumerate}

\section{Members}
\label{sec:members}
\subsection{Becoming an Individual Member}
\label{sec:indivmembers}
Individual Members of the DF Movement shall be as follows:
\begin{enumerate}[\hspace{0.5cm}(a)]
% TODO: Make this between 16 and 20 inclusive
\item \label{item:normalmember} a person who is aged between 16 and 20 and who has paid the annual subscription laid down from time to time by the Annual Conference of the Woodcraft Folk and signed up to the Aims, Principles and Programme of the Woodcraft Folk; or
% TODO: Let Lloyd know this is unconstitutional as per WcF constitution
% TODO: Ask GC to remove Treasurer bit as it's unnecessary
\item  \label{item:greyarea} a person who has taken on specific responsibility for the DF Movement or been elected on a committee, and is aged between 21 and 22, or is the Treasurer of the DF Movement aged 23, and who has paid the annual subscription laid down from time to time by the Annual Conference of the Woodcraft Folk and signed up to the Aims, Principles and Programme of the Woodcraft Folk.
\end{enumerate}

% TODO: This is structurally bad, suggest the following:
%An Individual Member of the DF Movement shall be a person who has paid the annual subscription laid down from time to time by the Annual Conference of the Woodcraft Folk and signed up to the Aims, Principles and Programme of the Woodcraft Folk and:
%\begin{enumerate}
%\item is aged between 16 and 20; or
%\item  has taken on specific responsibility for the DF Movement or been elected on a committee, %and is aged between 21 and 22, or is the Treasurer of the DF Movement aged 23.
%\end{enumerate}

\subsection{Membership rights}
\subsubsection{}
Members of the DF Movement organisation shall be entitled to the benefits set out in the Aims, Principles and Programme.
\subsubsection{}
Only members outlined in \ref{sec:indivmembers} shall have the right to vote on each motion and
candidates for election.
\subsubsection{}
All members of the DF Movement must support the Aims, Principles and Programme of the Woodcraft Folk.
\subsubsection{}
All members, that are required to do so, shall have completed a full child protection and safeguarding produces as laid out by General Council's regulations.

\subsection{Termination of Membership}
Membership shall not be transferable and shall cease on death. A Member shall cease to be
a Member of the organisation if:
\begin{enumerate}[\hspace{0.5cm}(a)]
\item they cease to be 16 to 20 in clause \ref{sec:indivmembers}(\ref{item:normalmember}) or;
\item they cease to be 21 to 22 in clause \ref{sec:indivmembers}(\ref{item:greyarea}) or;
\item they cease to hold a position of responsibility in clause \ref{sec:indivmembers}(\ref{item:greyarea}); or
\item they resign their membership in writing; or
\item their membership is terminated according to the Woodcraft Folk's Article 2.6 and associated regulations set out by the General Council.
\end{enumerate}

\section{Administrative Arrangements and Other Clauses}
\label{sec:admin}
\subsection{Conflict of Interest}
\subsubsection{}
% TODO: Ask GC to clarify "Committee" throughout Section A
Whenever a matter is to be discussed at a meeting or decided and a Committee member or other elected role has a Conflict of Interest in respect of that matter then they must:
\begin{enumerate}[\hspace{0.5cm}(a)]
\item withdraw from the debate and only remain for such part of the meeting as in the view of the other committee members is necessary to inform the debate;
\item not be counted in the quorum for that part of the meeting or decision-making process; and
\item withdraw during the vote and have no vote on the matter.
\end{enumerate}
\subsubsection{}
Members of the DF Committee and other elected roles shall abide by the Conflict of Interest Policy as laid down from time to time by the General Council.
\subsubsection{}
No member of the DF Committee or otherwise elected person may be employed or otherwise personally benefit from any transaction of the Woodcraft Folk without prior approval of the General Council.

\subsection{Minutes}
\subsubsection{}
The Secretary shall keep minutes of all:
\begin{enumerate}[\hspace{0.5cm}(a)]
\item appointments of officers made by the DF Movement;
\item resolutions of the DF Movement and of the committee; and
\item proceedings at meetings of the organisation and of the committee, including the names of the members present at each such meeting.
\end{enumerate}
\subsubsection{}
The minutes of Committee meetings must be kept for at least ten years from the date of the meeting, resolution or decision.
\subsubsection{}
The minutes of the meetings shall normally be considered open and shall be available to the Members on the organisation's website, except where those minutes relate to any reserved or confidential matters, including without limitation staff-related or disciplinary matters. The General Council may request all minutes, including those regarding confidential matters.

\subsection{Finances}
It is DF Committee as a whole that is responsible for the finances of the DF Movement with
the day-to-day operation in the hand of the Treasurer and the other authorised signatories.
\subsubsection{}
The Treasurer shall maintain accounting records and report on finances to each Committee meeting and quarterly to General Council.
\subsubsection{}
All expenditure shall be made only against agreed budgets set annually or for each event/activity.
\subsubsection{}
Procedures required by the General Council shall be followed including ensuring that all payments are signed by two signatories who are not related parties.
\subsubsection{}
Expenditure using restricted funds received will be accounted for clearly separate from that of other payments.
\subsubsection{}
The Treasurer shall be responsible for preparing annual accounts (Income and Expenditure Account and Balance Sheet) on a calendar year basis and submitting them within the time required by the General Council for consolidation as part of the charity as a whole.
\subsubsection{}
The Treasurer shall provide the person appointed to inspect or audit the accounts with the accounts, working papers, and all assistance that may be needed.

\subsection{Dissolution}
\subsubsection{}
If the committee decides that it is necessary or advisable that the DF Movement shall be dissolved, it shall call a General Meeting of the DF Movement by giving 7 days' notice in writing to each member and the General Council stating the terms of any resolution to be proposed.
\subsubsection{}
If it is decided at the Althing by a simple majority of those present and voting that the DF Movement shall be dissolved, the Committee shall wind up the DF Movement's affairs informing General Council of the resolution.
\subsubsection{}
Any assets remaining after the satisfaction of any proper debts and liabilities shall be given or transferred to The Woodcraft Folk, for development of work in the area in which the funds were raised. In the case of any restrictions of monies or assets held by the DF Movement the General Council shall determine who any funds should be distributed to in the fulfilment of the respective restrictions.
\subsubsection{}
A copy of the statement of accounts, or account and statement, for the final accounting period of the DF Movement should be sent to the Woodcraft Folk's registered office.

\subsection{Amending the Rules of Procedure}
\subsubsection{}
An Althing may amend the provisions of this deed, provided that:
\begin{enumerate}[\hspace{0.5cm}(a)]
\item no amendment may be made which limits of alters any powers or provisions relating to the General Council or Annual Conference of the Woodcraft Folk; and
\item no amendment may be made to clause \ref{sec:provisions} (Provisions), clause \ref{sec:members} (Members), or this clause \ref{sec:admin} (Administrative arrangements and other clauses) without the prior consent in writing of the General Council of the Woodcraft Folk; and
\item no amendment may be made whose effect breaks the Aims, Principles and Programme of the Woodcraft Folk, Policies of the Annual Conference or regulations of the General Council.
\end{enumerate}
\subsubsection{}
The committee must send to the General Council a certified copy of the Rules of Procedure noting any amendment made under this clause within two months of it being made for ratification by the General Council or their delegated officer.

\chapter{}
\section{DF Committee}
\subsection{}
There shall be a DF Committee (`Committee') composed of DFs and elected by DFs to oversee the implementation of the Movement's objectives as set out in article \ref{sec:objects}.

\subsection{Powers}
\subsubsection{}
To be the decision-making body for all DFs of the Woodcraft Folk subject to the rules and regulations of the Woodcraft Folk.
\subsubsection{}
To secure and administer funding for the work of the DF Movement and to invite and receive contributions provided that in raising funds Committee do not undertake any substantial permanent trading activities and conform to any relevant requirements of the law.
\subsubsection{}
To co-operate with other charities, voluntary bodies and statutory authorities operating in furtherance of the objects or of similar charitable purposes and to exchange information and advice with them.
\subsubsection{}
In order to achieve the above purposes, the DF business events (see section \ref{sec:business}) may from time to time establish sub-committees and working groups with delegated powers.

\subsection{Committee Roles and Officers}
\subsubsection{Primary Roles}
\label{sec:primaryroles}
\paragraph{} Committee shall be composed of a number of electable positions (or ‘primary roles’).

\paragraph{} The term of a primary role shall be two years, subject to the details outlined in \ref{sec:rolereplacement}.  The exceptions to this are the Chair, Events, Treasurer and Shadow Treasurer roles, whose details are given in sections \ref{sec:chair} and \ref{sec:treasurer}

\paragraph{} The primary roles shall be the following:

\begin{itemize}
\item Affiliations Representative
\item Campaigns Representative
\item Chair of Committee
\item Communications Representative
\item Districts Representative
\item Events Representative
\item Fundraising Representative
\item MEST-UP Coordinator
\item Secretary of the Movement
\item Shadow Treasurer (If this is applicable according to the rules set out in \autoref{sec:treasurer})
\item Training Representative
\item Treasurer
\item Lay Member 1
\item Lay Member 2
\end{itemize}

\subsubsection{Secondary Roles}
\label{sec:secondaryroles}

\paragraph{} In addition to the primary roles outlined in \ref{sec:primaryroles}, Committee will elect, from amongst their number, individuals to hold the following positions (`secondary roles') in addition to the primary role they are incumbent in.

\paragraph{} The term of a secondary role shall be one year.

\paragraph{} The secondary roles shall be the following:

\begin{itemize}
\item Sustainability Representative
\item Vice Chair of DF Committee
\item General Council Representative 1
\item General Council Representative 2
\item Chair Carer
\end{itemize}

\paragraph{} The two General Council Representatives will sit on General Council for a term of 1 year.

\subsection{Election of Committee Members}
\label{sec:election}
\subsubsection{}
Elections for primary roles on Committee shall be held at Althing (see section \ref{sec:althing}), and nominations may be made up until the point at which hustings for the positions begins.

\subsubsection{}
All candidates nominated for positions on Committee must be members of the Woodcraft Folk and DFs, as set out above.

\subsubsection{}
Committee terms begin and end at Old/New.  For example, a two-year term will begin at the Old/New (see \ref{sec:oldnew}) immediately after which they were elected until the Old/New held two years later.

\subsubsection{}
The election system to be used at Althing shall be as detailed in the Democratic Procedures document.  The voting system shall be AV for single-role elections, and STV for multi-seat elections.

\subsubsection{}
\label{sec:moreroles}
The same person can fill more than one role, with the exception of the safeguarding representative.

\subsubsection{}
\label{sec:jobshares}
With the exception of the safeguarding representative, a single role on committee may held by two people, who must run and be elected as a pair. This allows for the workload and responsibility to be shared between two people. However as both people are still only filling one role on committee, only one person may be expensed to business events, and only hold one vote between them on committee matters. The decision of which person to expense is at the discretion of those holding the role.

\subsubsection{}
If a position on Committee remains or becomes vacant, Committee may co-opt at their discretion a member of the DF Movement to fill the position.

\subsection{Joining and Leaving Committee}
\label{sec:joiningleaving}

\subsubsection{Vacation of a Role}
A person may no longer serve on Committee if:
\begin{enumerate}[\hspace{0.5cm}(a)]
\item they cease to be a member of the organisation; or
\item they stand down by notice to the DF Movement; or
\item they lose a vote of no confidence as set out in \ref{sec:noconfidence}
\end{enumerate}

If a person holds more than one role, each role will be assessed individually when following the above criteria.

\subsubsection{Vote of No Confidence}
\label{sec:noconfidence}
\paragraph{} Any DF may call a vote of no confidence on a role on committee; this vote will be taken by Committee (excluding the member in question) and will require an 2/3rds majority vote to pass.
\paragraph{} A vote may be held at any time as long as the Committee role in question was given at least 10 days’ notice of the vote, and is offered to present their case to Committee either in person or in writing before the vote. Every member of Committee must be involved in the discussion and the vote.
\paragraph{} In the case of physical or mental incapacity, lack of attendance, or lack of report production to Committee meetings twice in a row, a vote of no confidence shall take place automatically. This action is only to be taken when internal Committee support structures have failed.

\subsubsection{Replacement of a Role}
\label{sec:rolereplacement}
\paragraph{} If any vacancies arise on Committee, Committee may co-opt any member of the Movement to fill the vacancy. This will require a 2/3rds majority decision to pass.
\paragraph{} If a person holding a 2-year role, with the exception of the Campaigns Rep and either Lay Member, stands down mid-term at an Althing, an election shall be held in the usual way for the role, and the role will be held by the winner of the election for a further 2 years.
\paragraph{} Since campaigns are elected for 2 years at a time, if a new Campaigns Representative is elected mid-term, they will only serve up until the end of the campaign underway when they were elected, i.e. for 1 year.
\paragraph{} This does not apply to Lay Members, as there should always be one Lay Member elected at each Althing. Therefore if a Lay Member stands down mid-term, the person elected to replace them will only serve until the end of the term of the person they are replacing, i.e. for 1 year.

\subsection{Role of the Chair}
\label{sec:chair}

The role of the Chair is unique on Committee as its purpose is to serve Committee and not the Movement at large, and should therefore be electable by Committee, rather than the Movement.  However, the role is too large to be a secondary role.  Thus, the following two-stage election shall be implemented.

\subsubsection{}
At Althing, an STV election shall be held for the position of Chair.  All candidates coming beneath RON shall be discarded.

\subsubsection{}
Committee shall have a closed meeting at Althing, during which they shall hold an election by AV for the position of Chair.  The candidates for this closed election shall be the pool of remaining candidates from the previous stage.

\subsubsection{}
For clarity, at this election, Committee may elect RON, leaving the position vacant.

\subsection{Role of the Treasurer}
\label{sec:treasurer}
\subsubsection{}
The Shadow Treasurer and Treasurer role are both uniquely different to other Committee roles. It has been recognised that the role of Shadow Treasurer and Treasurer are particularly challenging and important to the movement. This system set out below has therefore been adopted and supersedes the provisions set out above.
\paragraph{}
The Shadow Treasurer is elected for a term of one year from the floor of Althing. As with other roles, the incumbent Shadow Treasurer may re-stand.
\paragraph{}
The Treasurer is elected for a term of one year by the floor of Althing. Only previous Shadow Treasurers (or the incumbent or previous Treasurers) may stand.  This will mean that any Treasurer elected at an Althing will have already been a Shadow Treasurer at some point.
\paragraph{}
The Shadow Treasurer will work with the Treasurer over the next year to learn the skills necessary to carry out the role effectively.
\paragraph{}
Any candidate for Shadow Treasurer must be under 20 years old at Althing, so that after completing a term as Shadow Treasurer, they may still stand for the position of Treasurer.
\paragraph{}
The Treasurer shall produce quarterly financial reports of the DF accounts to be sent to both DF Committee and General Council.
\paragraph{}
\label{sec:treasurerreport}
Before the election of a new Treasurer and Shadow Treasurer, Committee shall submit a report to Althing on their success in the role which shall be included in the minutes. This is to ensure that if that person stands for the role of Treasurer in future years, an informed choice may be made by the floor of Althing.
\subsubsection{No Confidence in the Treasurer}
\label{sec:treasurernoconfidence}
\paragraph{} If the Treasurer position becomes vacant, by a vote of no confidence or otherwise, the Shadow Treasurer shall become the Treasurer and a new Shadow Treasurer shall be co-opted by DF Committee in the usual way (\ref{sec:rolereplacement}).
\paragraph{} If the Shadow Treasurer is unable to fully take on the role of Treasurer, an additional Shadow Treasurer can be elected for one year. The Treasurer of Woodcraft Folk and DF Committee will support both Shadow Treasurers in their training for the year and will help write the yearly report for our financial audit.
\paragraph{} The decision to either hand over or take back power of the accounts will be made by DF Committee.

\subsection{Role of the MEST-UP Coordinator}
\label{sec:mestup}

The role of the MEST-UP Coordinator presents similar issues to that of the Chair as its purpose is to serve the MEST- UP team, as such the election process shall be in two parts as stated below.

\subsubsection{}
At Althing an STV election shall be held for the position of MEST-UP Coordinator. All candidates coming below RON shall be discarded.

\subsubsection{}
A closed meeting of MEST-UP involving a minimum of 11 reps or half of those trained and of DF age, whichever is fewer, will be held during which they shall have an AV election for the position of MEST-UP Coordinator. The candidates for this closed election shall be the pool of remaining candidates from the previous stage.

\subsubsection{}
For clarity, at this election, MEST-UP may elect RON, leaving the position vacant.

\section{Supervisory Roles}
\subsection{Accessibility Representative}
\subsubsection{}
The role of accessibility representative is singular in many ways. It is a vitally important role and unlike many other roles needs access to committee meetings but without being on committee. The point of the role is to ensure that all of the DF movement is accessible, including events and committee, if the role were to be on committee being impartial would be near impossible. However, they do need access to part of committee to ensure that it is being as accessible as possible.
\subsubsection{}
The election of the Accessibility Rep, like many other roles, is done by the movement at large. Because of the importance of the role, as well as being able to RON candidates, members can also choose to Veto candidates. If a candidate is Veto-ed, they are automatically disqualified, regardless of how many votes they have. A Candidate only requires one Veto to be disqualified – the Veto is anonymous and does not need to be explained or justified.
\subsubsection{}
Accessibility may be held at the same time by someone holding a non-committee role, with the exception of shadow events, but cannot be held by someone serving on DF committee.


\section{Non-Committee Roles}
\subsection{}
There shall be several elected roles that are not on committee composed of DFs and elected by DFs to oversee areas of the movement’s objectives that do not fall within the remit of DF Committee.
\subsection{Role Terms and Titles}
\subsubsection{}
The term of all non-committee roles shall be one year except for the two GC representatives, whose term-length details are given in \ref{sec:noncomgeneralcouncil}
\subsubsection{} The primary roles shall be the following:

\begin{itemize}
\item Zine
\item Web Fairy
\item First Aid Fairy
\item Venturer Committee and Venturer Camp Liaison
\item International Opportunities
\item Workers' Beer Representative
\item General Council (Even)
\item General Council (Odd)
\item Podcast Coordinator
\item London Liaison
\item Shadow Events
\item Common Ground (International Camp 2020) Board Representative
\end{itemize}

\subsection{Election of Non-Committee Roles}

\subsubsection{}
Elections for Non-Committee roles shall be held at Althing (see \ref{sec:althing}), and nominations may be made up until the point at which hustings for the positions begins.

\subsubsection{}
All candidates nominated for Non-Committee positions must be members of the Woodcraft Folk and DFs, as set out above.

\subsubsection{}
Non-Committee terms begin and end at Old/New.  For example, a two-year term will begin at the Old/New (see \ref{sec:oldnew}) immediately after which they were elected until the Old/New held two years later.

\subsubsection{}
The election system to be used at Althing shall be as detailed in the Democratic Procedures document.  The voting system shall be AV for single-role elections, and STV for multi-seat elections.

\subsubsection{}
The same person can fill more than one role.

\subsection{Role of General Council Representatives}
\label{sec:noncomgeneralcouncil}
\subsubsection{}
Unlike other non-committee roles the General Council reps shall both be two year roles, elected alternately, following the same election procedures as all other non-committee roles. The elections for these roles shall correspond to their title; General Council Even shall be elected at Althings in even years (i.e 2020, 2022, etc) and General Council Odd shall be elected at Althings in Odd years (i.e 2019, 2021, etc)

\subsection{Joining and Leaving a Role}
\subsubsection{Vacation of a Non-Committee Role}
A person may no longer remain in a role if:
\begin{enumerate}[\hspace{0.5cm}(a)]
\item they cease to be a member of the organisation; or
\item they stand down by notice to the DF Movement; or
\item they lose a vote of no confidence as set out in \ref{sec:noncomnocon}
\end{enumerate}

\subsubsection{Vote of No Confidence}
\label{sec:noncomnocon}
\paragraph{} Any DF may call a vote of no confidence on any Non-Committee role; this vote will be taken by Committee and will require an 2/3rds majority vote to pass.
\paragraph{} A vote may be held at any time as long as the role in question was given at least 10 days’ notice of the vote, and is offered to present their case to Committee either in person or in writing before the vote. Every member of Committee must be involved in the discussion and the vote.

\subsubsection{Replacement of a Role}
\paragraph{} If any vacancies arise within non-committee roles, Committee may co-opt any member of the Movement to fill the vacancy. This will require a 2/3rds majority decision to pass. This person will hold the role until the next Althing.
\paragraph{} If a General Council position becomes vacant mid-term the role shall be put up for election at the next Althing regardless of year, and will last for only a one year term to resume the intended election order of alternate General Council representative elections.


\section{Business Events}
\label{sec:business}
\subsection{Althing (Annual General Meeting)}
\label{sec:althing}
\subsubsection{}
Each year, there shall be an Althing, open to all DFs, which will be the Movement's Annual General Meeting.
\subsubsection{}
In choosing the date for Althing, Committee shall ensure that there is sufficient time to allow the execution of motions that call for the submission of motions to the Woodcraft Folk's Annual Gathering.
\subsubsection{}
All DFs attending Althing shall have one vote on each motion and may vote in elections. All members of Committee  who are in the `Grey Area' (clause \ref{sec:indivmembers}(\ref{item:greyarea})) also have one vote on each motion and may vote in elections. No other attendees of Althing may vote on motions or in elections.
\subsubsection{General Council Representatives}
Althing shall elect 2 representatives to serve on the Woodcraft Folk's General Council.
\subsubsection{The general business of Althing shall be as follows:}
An in-depth description of motion and musing procedure, as well as an outline of the discussion formats to be used can be found in the Democratic Procedures document.

\begin{enumerate}
\item Ratification of minutes of the previous Althing.
\item Reports from Committee members describing their activities over the past year.
\item A financial report.
\item Discussion of and voting on motions.
\item Discussion of musings.
\item The Treasurer and Shadow Treasurer report from Committee, as detailed in \ref{sec:treasurerreport}
\item Nominations, hustings and elections for places on Committee.
\item Any other business.
\item Workshops, discussions and other sessions of interest to members.
\end{enumerate}

\subsubsection{Althing Organisation}
% TODO It would make more sense to put this in the DPs since it's practical rather than to do with the nature of DFs.
It is the responsibility of the Chair and Vice Chair of DF Committee to organise the practicalities of Althing.


\subsection{Old/New}
\label{sec:oldnew}
\subsubsection{} Annually, not more than two months after Althing, DF Committee shall hold a closed meeting called `Old/New'.

\subsubsection{} The business of Old/New will include the election of secondary roles (see \ref{sec:secondaryroles}).

\subsection{Things (General Meetings in a particular Region)}
\subsubsection{}
Each year, there shall be three DF business events called `Things', the dates and venues for which shall be announced by Committee.
\subsubsection{}
The location of Things shall rotate by groups of regions and nations of the UK:

If Althing falls in a year which is a multiple of 3 (e.g. 2013, 2010, 2007), the Things held in the year after that Althing will be in:
\begin{itemize*}
\item South West
\item Scotland
\item Midlands
\end{itemize*}

If Althing falls in a year which is one more than a multiple of 3 (e.g. 2011, 2008, 2005), the Things held in that year will be in:
\begin{itemize*}
\item London
\item North West
\item South East
\end{itemize*}

If Althing falls in a year which is two more than a multiple of 3 (e.g. 2012, 2009, 2006), the Things held in that year will be in:
\begin{itemize*}
\item North East
\item Wales
\item Eastern
\end{itemize*}

\subsubsection{The general business of Things shall be as follows:}
\paragraph{}
Reports from Committee members describing their activities since the last business event, be it a Thing or Althing.
\paragraph{}
A financial report.
\paragraph{}
Discussion of forthcoming events, campaigns, etc. 
\paragraph{}
Any other business.
\paragraph{}
Discussions and workshops about issues that are of interest to DFs. 

\subsection{Quorum}
\label{sec:quorum}
No business shall be transacted at any Althing or Thing unless a quorum of voting
members are present.
\subsubsection{Althing} The quorum for decisions taken at Althing shall be thirty-one registered DFs, or twice the number of Committee members present plus one, whichever is the lesser. The minimum quorum for Althing shall be the number of committee roles total.
\subsubsection{Things} The quorum for decisions taken at a Thing shall be half the number of people present at the Thing plus one. Quorum is required for all non-closed Committee discussions at Things.

\end{document}
