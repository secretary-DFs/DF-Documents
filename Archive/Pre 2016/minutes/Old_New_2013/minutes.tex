% !TEX TS-program = xelatex
% !TEX encoding = UTF-8

% This is a simple template for a XeLaTeX document using the "article" class,
% with the fontspec package to easily select fonts.
\documentclass[a4paper, 11pt]{article} % use larger type; default would be 10pt

\usepackage[british]{babel}
\usepackage{fontspec}
\defaultfontfeatures{Mapping=tex-text}
\setmainfont[Ligatures=TeX]{Open Sans}

\usepackage[scale=0.9]{geometry}
\usepackage[parfill]{parskip} % Activate to begin paragraphs with an empty line rather than an indent
\usepackage[hidelinks]{hyperref}

\usepackage{graphicx} % support the \includegraphics command and options

\usepackage{multirow}
\usepackage{setspace}
\usepackage{footnote}

\renewcommand{\arraystretch}{1.2}

\newcommand{\myquote}[2][Anonymous]{
``#2''

\hspace{15pt} --- #1
}

\title{Old/New 2013 Minutes (Draft)}
\author{The DF Movement}
\date{\today}

\setcounter{tocdepth}{2}

\begin{document}
\maketitle

\tableofcontents

\section{Roles}

\begin{savenotes}
\begin{center}
	\begin{tabular}{ l  l  l } \hline \hline
	\textbf{Committee Role}		& \textbf{Old}			& \textbf{New} \\ \hline \hline
	Affiliations				& \multicolumn{2}{c}{Ellie Ferguson} \\ \hline
	Campaigns				& Rosie Booth 			& Saskia Neibig \\ \hline
	Communications			& \multicolumn{2}{c}{Louise Delmége} \\ \hline
	Districts					& [new role]			& Emily Locke \\ \hline
	Events					& Ruori McIntyre 		& Lily Bowler \\ \hline
	Fundraising				& Joe F. Bowler (Jay) 	& [role dissolved] \\ \hline
	International				& Imogen Smith 		& Anna Rathbone \\ \hline
	MEST-UP\footnote{MEST-UP was made a Committee role this year.}	& \multicolumn{2}{c}{Issy Rose (re-elected)} \\ \hline
	Projects					& Emily Connor 		& Joe Flannagan \\ \hline
	Training					& Saskia Neibig 		& Naomi Wilkins \\ \hline
	Treasurer					& [unfilled]\footnote{RON was elected at Althing, and financial responsibility fell to General Council as per section B.1.7.2 of the Constitution.}	& Ellie Mestel \\ \hline
	Secretary					& \multicolumn{2}{c}{Joe MacMahon (Kess)} \\ \hline
	Shadow Treasurer			& Ellie Mestel 			& Nick FitzGerald \\ \hline
	Chair of Regional Council		& Emma Jagne 		& [role dissolved] \\ \hline
	Lay Member (Odd)			& Jess Poyner			& Josh Hope-Collins \\ \hline
	Lay Member (Even)			& Henry Care\footnote{Henry stood down early as he turned 21.}	& Ryan Hilson \\ \hline
	Chair					& Jess Poyner 			& Imogen Smith \\ \hline
	Vice Chair				& Saskia Neibig 		& Ryan Hilson \\ \hline
	\multirow{2}{*}{General Council Reps}	& \multicolumn{2}{c}{Louise Delmége (re-elected)} \\ \cline{2-3}
							& Leonor Worssam\footnote{Elected from Regional Council}	& Anna Rathbone \\ \hline
	Chair Carer				& [new role]			& Issy Rose \\ \hline
	Sustainability				& Emily Connor 		& Lily Bowler \\ \hline
	Staff Liaison				& Saskia Neibig 		& Saskia Neibig \\ \hline \hline
	\end{tabular}
\end{center}
\end{savenotes}

\begin{savenotes}
\begin{center}
	\begin{tabular}{ l l l } \hline \hline
	\textbf{Non-Committee Role}		& \textbf{Old}			& \textbf{New} \\ \hline
	Webfairy						& \multicolumn{2}{c}{David Moore} \\ \hline
	Zine Editor					& Sophie Slater		& Alec Mezzetti \\ \hline
	Venturer Committee Liaison		& Lily Bowler			& Lily MacTaggart \\ \hline
	First Aid						& Mez Griffin			& Angus Wood \\ \hline
	London Liaison					& [new role]			& Ruairí O'Boyle \\ \hline
	Workers' Beer					& Sophie Slater		& Toby Attril \\ \hline
	Share Co-ordinator				& Sam Sender			& [role dissolved] \\ \hline
	\multirow{2}{*}{General Council Reps}		& Lily Bowler	& Naomi Wilkins \\ \cline{2-3}
								& Sam Sender			& [unfilled] \\ \hline \hline
	\end{tabular}
\end{center}
\end{savenotes}

\section{Chair Election}

\subsection{Hustings}

\begin{onehalfspacing}
\subsubsection{Anna Rathbone}
I'm not very good at writing hustings\ldots

But I'm quite good at meetings and discussing things.  I haven't had much experience of chairing discussions, but I'd really like to learn how (although I can do that without being chair).

Stuff you might like to know about my life that might come up in discussion:

\begin{itemize}
\item I'm in my first year doing French and Russian (ab initio) at Bristol Uni.  So doesn't count towards my degree, but it's quite intense nontheless.
\item I'm from a little district called West Coventry.
\end{itemize}

Not many people within DFs knows me very well because I haven't been to very many events --- make of that what you will.

I enjoy organising stuff and making things happen.  I also don't mind listening to people and working out the best ways to resolve issues and problems.  I'm a busy person, but because of that I'm also quite an organised one.

It's hard to write a hustings in 10 minutes, so I don't really know what else to say.  But now there are 3 candidates for chair --- yay!  Democracy!

\subsubsection{Imogen Smith}
Hello!

My name is Imogen. Most of you know me, but for those of you who don't, I'm 19, I live in Bradford and I'm not at school OR university, I am currently searching for a job. This means that I have an awful lot of free time (too much for my liking) to give. I would love to be about to have the job of chair to occupy my time! I have loads of time to give to Dfs and can't think of a better way to spend this year.

I have been on committee as the International Opportunities Officer for the past two years. I've really enjoyed this post and have learnt a lot about how committee works. I have a solid understanding of the importance of having a strong committee, who communicate a lot as well as a successful support system for committee members. Its really important to make sure that committee members don't feel over loaded and create a culture where we are aware when things are going wrong and can offer support. Its also vital that everyone on committee feel comfortable asking for help and I want to create a culture where everyone feels happy and valued as a committee member. I believe that this will help committee and DF's to be more efficient.

I have done a small amount of facilitating before. Whilst I am far from being an expert, I would be grateful to have to opportunity to strengthen this skill and become a better and more confident facilitator. I feel I would make a successful facilitator because I rarely have strong opinions on specific DF matters. What matters more to me is that a decision is make fairly and that action then happens based on that decision. I would not have to hold myself back in discussions as I usually sit back and listen anyway! I am good at working out the feel of a discussion and can digest other peoples points effectively.

I have been involved with Woodcraft consistently since the age of 5. Woodcraft has shaped me as a person and as such, my `aims and principles' are pretty much tied in with Woodcraft and DF's aims and principles. I would be comfortable in representing the view of the District Fellows movement to outside organisations when required.

I took on the role of coordinator for DF camp 2011 so have experience of being sober and responsible at events. I am able to take charge of situations if they arise and am aware of safeguarding issues. And can make quick and sensible decisions. Whilst I prefer to work as part of a team, I am capable and willing to step up and take charge when necessary.

I will available to attend all Things and events throughout the year.

I think of myself as a very approachable and friendly person. I am good at talking to people and enjoy a good catch up on a regular basis with committee members. Therefore I am sure that I would enjoy the task of pastoral care within committee, checking up on people and making sure everyone is happy and on task wouldn't be a chore for me. I find the idea of hearing about what everyone has been doing in their roles really exciting. Once I am in the know and aware of what everyone on committee is up to I plan to make DF's more transparent, I think that everyone should know exactly what is going on in committee and all the great things we are doing. I also plan to take on the role of the `Hype rep' that was discussed at Eastern Thing, so that other volunteers also feel valued. This is all part of my vision for Dfs, I want to see Dfs grow, I want bigger events and better attended Things, I want to see fuller agendas and more contribution from the wider movement on the matters we discuss, I want to see the military out of school and MEST-UP into schools, I want everyone to know about the amazing work we do! I want Dfs to be bigger, more transparent and more inclusive.

Vote Imogen for Chair!

\subsubsection{Jack Yeo}

I have been in Woodcraft since I was 13, and active in the wider DF movement since 17. It has been such a uniquely wonderful experience that it has filled me with a deep sense of gratitude and responsibility to the committees that have guided the organisation through that period, to take its values to heart and live them through my actions, and to actively give back whenever I can. It is with this rationale that I present my hustings for the role of Chair of DF Committee:

I am no stranger to being on committees and dealing with everything that entails. During my first year of University I served both as vice-chair and treasurer on my Halls Committee, a reluctant group, with the chair and myself being the only regularly active contributors. This was my first experience of going above and beyond the letter of my job description, and actually, I flourished, learning a lot in a short period of time and putting on several large events including a Christmas Ball for 200 people, effectively as a group of two. The variety of skills this required, as well as the workload, speak for themselves. From there I went on to Welsh Council for Woodcraft Folk. This gave me experience of working with a more cohesive committee, as well as gaining Woodcraft specific know-how. Whilst it was undoubtedly a more pleasant experience, it is just as important knowing how to work with others and get the most out of each other as it is taking on jobs of others who are unable to fully perform their duties. All this, I carried into my role as Treasurer of my University's LGBT society, where I am currently in my second year-long term, a role where, despite working with a brilliant group of people, the sheer amount that we did and the unpredictability of it meant that, again, I had to break out of my defined role. This also gave me more chair-specific experience, as being treasurer made me de-facto vice chair, a role I had to take on many occasions. But even this was dwarfed by my experience of running the Donkey's Head Café at Venturer Camp, an incredibly rewarding but incredibly tiring, 16 hour daily marathon that, despite extensive planning constantly required me to be doing things that nobody could have predicted. This included things I had never done before and had to take the initiative to either figure out or just have a go and hope a mix of common sense and problem solving skills would lead to an agreeable outcome, which in most cases, it did. Despite it being difficult, I really enjoyed throwing myself into these new situations, relishing the challenges and the chances to prove myself to be adaptable, hard working, and a quick learner, whilst contributing to the success of something that I cared so much about, so that others could benefit from it.
I love doing whatever it takes to make something that I am passionate about happen, whether it means doing the most unexpected and ridiculous of jobs, or simply being “that person” to put forward an unpopular, but necessary-to-consider viewpoint in a discussion. And there are few things I am more passionate about than the DF Movement. I want to do well by this organisation and give back to it. I am passionate about making sure it is accessible to as many people as possible, and about being as fair to as many viewpoints as possible. Whilst I acknowledge I am seen as an opinionated person, my opinion on any given issue is insignificant compared to my deeply held opinion that, in an organisation such as this, all voices should be heard and no opinion should be disregarded or given unfair weight. Having had to administrate the Facebook page for the Leeds LGBT Society, I have had to bite my tongue on many an occasion in order to balance or lighten the tone of the discussions that result on that page, which, unsurprisingly, given that we are effectively a society for people who have absolutely nothing in common, can often be as heated as those seen on the DF page. 
 I would endeavour, not just on committee but within the wider membership, to pro-actively include the voices of those whose circumstances make it more difficult to speak out. Ultimately, as well as being pro-active, the time comes when we also just need to listen to the people who feel this movement to be inaccessible. As somebody who has been a member of this organisation whilst subject to various accessibility needs of my own, I feel I am uniquely in a position to empathetically facilitate this, whilst also not letting my own needs affect my role, as we discussed at Althing (The breathing mask has lead to significant improvements as predicted).

Despite thinking it is of crucial importance that committee actively hold themselves to account, I also know from experience that being on committee is a tough job, and one where you only ever get negative feedback – people rarely bother to thank you for what you do right but will make damn well sure you hear about it if you do the slightest thing wrong! I myself have been guilty of this, but what I have learned is that inter-committee support and love is essential to the functionality and well-being of its members, ensuring we are all happy and continue to want to be actively involved in making this movement the best thing it can be. To this end, I would endeavour to never disregard the social aspect of our business meet-ups, as I feel downtime when we can interact as friends and not committee members will be essential to our mental health and ability to do our jobs. I would also make sure to minute time in meetings for “Committee Feels”, an open floor for people to air grievances and personal issues that need discussion in a productive way. As well as this, I would facilitate the implementation of an inter-committee structure for discussion of private welfare matters and conflict resolution. The nature of this structure would depend on who would want to be involved and how. This is something that in particular I would talk to the MEST-UP rep about, but even if that structure ended up just being me, I feel I would be able to deal with this using the welfare advice-giving and mediation skills that are a necessary aspect of my position on LGBT committee – I am already in a role where I must be ready to give welfare any time, any place, from any member who approaches me, so this would not be an extra stretch.

In summary, I am a passionate candidate with a strong belief in making sure every voice is heard and that committee welfare remains a priority, with the experience needed to make this, along with the day-to-day aspects of chairing committee from a non-hierarchical viewpoint, a reality; and throwing myself head first into absolutely anything that needs doing to continue to make this organisation the fantastic thing that it is in any way that I can.

\end{onehalfspacing}

\subsection{Results}

Imogen was elected.

\section{Dates}

\subsection{Business}
\begin{description}
\item[South West Thing] 6th--8th December
\item[Midlands Thing] 7th--9th March
\item[Scottish Thing] 11th--13th July
\item[Althing] 29th--31st August
\item[Next Thing] 5th--7th December
\end{description}

\subsection{Social Events}
\begin{description}
\item[Winter Wonderland] 27th--30th December
\item[Valentine's Hostel] 14th--16th February
\item[Spring Awakening] 12th--16th April
\item[DF Camp] 30th July--6th August
\end{description}

\section{Motions}
\subsection{Motion 13}
\subsubsection{Amendment 1}
Not lapse motion from 2010 about leadership training.

(Naomi Wilkins)

Accepted by proposer.

\subsubsection{Amendment 2}
Keep the Winter Wonderland being booked 15 months in advance motion (motion 2, 2008).

(Imogen Smith)

Withdrawn.

\subsubsection{Amendment 3}
Lapse the Span That World with Music motion.

(Joe Bowler)

Amendment clearly passed.

Motion passed unopposed.

\subsection{Late Motion 8}
\subsubsection{Amendment 1}
Replace the text of the motion with:

\begin{quote}
Nick FitzGerald (incoming Shadow Treasurer) should look into accounting software.
\end{quote}

(Joe Bowler)

Accepted by proposer.

Motion withdrawn.

\section{Communications}
\subsection{Technical Workshop}
Louise held a workshop on how to use the website, Google Docs and the Google Mail accounts.

\subsection{Online Communications Workshop}

\begin{itemize}
	\item How to deal with difference of opinions within committee:
	\begin{itemize}
		\item We can state that there was discussion, but don't mention what was said by whom.
		\item Sometimes we should have a united front as committee.  Chair should decide this.
	\end{itemize}
	\item Don't be a dick.
	\item When talking to other DFs, make sure it's clear whether you are talking as a committee member or saying your personal opinion.
	\item Everyone can tell people to chill out, maybe take a break and come back to a discussion later if it is getting heated.
	\item Don't use all-capitals; grammar is important.
	\item Be aware of tone and language online.
	\item Sarcasm doesn't always translate.
	\item Move conversations from Facebook onto email if you want as it can get stressful.  Keep things off social networks where possible.
	\item Keep the response time to instant messages the same as the response time to email.
\end{itemize}

\section{Secondary Roles}
\begin{center}
	\begin{tabular}[H]{l l l} \hline \hline
	\textbf{Role}							& \textbf{Nominees}	& \textbf{Elected} \\ \hline \hline
	\multirow{3}{*}{Vice Chair}				& Anna Rathbone		& \multirow{3}{*}{Ryan Hilson} \\
										& Josh Hope-Collins		& \\
										& Ryan Hilson			& \\ \hline
	Sustainability							& Lily Bowler			& Lily Bowler \\ \hline
	\multirow{2}{*}{Staff Liaison	}			& Josh Hope-Collins		& \multirow{2}{*}{Saskia Neibig} \\
										& Saskia Neibig		& \\ \hline
	\multirow{3}{*}{General Council Reps}		& Anna Rathbone		& \multirow{3}{*}{Anna Rathbone \& Louise Delmége} \\
										& Joe Flannagan		& \\
										& Louise Delmége		& \\ \hline
	\multirow{3}{*}{Chair Carer}			& Issy Rose			& \multirow{3}{*}{Issy Rose} \\
										& Lily Bowler			& \\
										& Ryan Hilson			& \\ \hline \hline
	\end{tabular}
\end{center}

\section{Actions arising from Althing motions}
\begin{center}
	\begin{tabular}[H]{l l l} \hline \hline
	\textbf{Motion}	& \textbf{Actioned Role}	& \textbf{Details} \\ \hline \hline
	1				& Secretary			& Put on STW; add Naomi's flowchart. \\
	3				& Secretary \& Louise	& Liaise and make better system, and let DFs know. \\
	5				& Treasurer \& VC Liaison	& Pay the money and do the report yo. \\
	9				& Cams., Comms., Intl., \& Affiliations & Making sure things are publicised and affiliated to. \\
	11				& Communications \& Events	& Advertise and put in events pack. \\
	Late 1			& Treasurer			& Pay the moneys. \\
	13				& Secretary			& Fix all that shit up. \\
	Late 8			& Shadow Treasurer	& Look into accounting software. \\
	\hline \hline
	\end{tabular}
\end{center}

\section{Other actions}

\section{Expectations}
\subsection{Training (Naomi)}
\begin{itemize}
	\item Help with MEST-UP training
	\item Small events training sessions, e.g. WW
	\item All that first aid stuff
	\item Advertise on STW about training
	\item Steward training
	\item Conflict resolution
	\item Call Saskia (previous Training) if questions
	\item Help organise Cams\&Comms training weekend with Louise and Saskia
	\item Committee training
	\item Have booking forms for training events on social events (and other training events)
	\item Online/distance training, e.g. KPs having food hygiene training
	\item Talk to Nick about money
	\item STEM cell training
\end{itemize}

\subsection{MEST-UP (Issy)}
\begin{itemize}
	\item Get funding for some training --- talk to Naomi and Saskia
	\item Lots of cool official style training in how to do condom demonstrations and chlamydia testing
	\item Get MEST-UP into festivals and schools
	\item Make it better on DF events --- accessible, obvious
	\item Midlands mediation network to come to training in November
	\item Training on dealing with people with learning difficulties and non-sober people
	\item Getting a massive MEST-UP team
	\item Talk to Joe\footnote{All of them.} cos he's cool.
	\item Making people aware of initial training
	\item Talk to Nick about money
	\item Re-engage with old MEST-UP reps
	\item Talk to Imogen about MEST-UP in schools
	\item Mediation workshop at a Thing
	\item Venturer Committee having mediation training --- talk to Lily M
	\item Clarify the kinds of MEST-UP training available
	\item Co-operation between MEST-UP and first aid
	\item Talk to Ryan about WW
\end{itemize}

\subsection{Even Lay Member, Vice Chair (Ryan)}
\begin{itemize}
	\item Supporting everybody
	\item Doing lots of little things
	\item Be cried on by Angus
	\item Make posts on STW
	\item Talk to Imogen every day
	\item Poke people when they need it, esp. Nick
	\item BE GOOD
\end{itemize}

\subsection{First Aid (Angus)}
\begin{itemize}
	\item Get right on all that training shizzle
	\item Help people with stewarding things
	\item Get some help with funding for training events
	\item Keep first aid books
	\item Co-ordinate shift rotas with MEST-UP better
	\item Co-ordinate the first aid kits
	\item Rock That Hi-Vis, but don't be a fascist
	\item Kiss Josh
\end{itemize}

\subsection{Secretary (Joe Mahanansjfsldkfjsldkfj;erkmgmxvlkjsfdkljfsdkjgfkjh)}
\begin{itemize}
	\item Secretary all over the budget
	\item Make documents accessible on STW
	\item Help Imogen with the `What's changed from 2013' post on STW
	\item Actually do all the minutes
	\item Help people understand procedures and policies
	\item Motions at Althing in good order
	\item Carry on being banning
	\item No paper cuts
\end{itemize}

\subsection{Communications (Louise)}
\begin{itemize}
	\item Training people on how to use the website
	\item Teach lots of people the basics of lots of things to avoid a core group doing everything
	\item Venturer Committee (and venturers in general) getting training in websites
	\item Communicating with venturers about things going on
	\item Talk to Lily B about events and advertising them
	\item Advertise international events
	\item Don't overload yourself
\end{itemize}

\subsection{Venturer Committee Liaison (Lily M)}
\begin{itemize}
	\item Get money
	\item Support venturers, in particular with long-term projects
	\item Talk to Jack Brown
	\item Feed back lots so we can let DFs know about what cool stuff venturers are doing
	\item Talk to Naomi if you want any training
	\item Make sure venturers know that DFs exist and have lots of help to give
\end{itemize}

\subsection{Odd Lay Member (Josh)}
\begin{itemize}
	\item Promote business events to DFs at large
	\item Do lots of fundraising
	\item Picking stuff up from the office
	\item Talk to Anna about funding for international delegation
	\item Ask around for jobs, rather than just waiting for jobs to come to you
	\item Do jobs for Secretary and Training if they don't have time
\end{itemize}

\subsection{Campaigns (Saskia)}
\begin{itemize}
	\item Make sure there's a MOOS campaing workshop at every event
	\item Broaden age-groups involved in the campaign
	\item Get into schools with other organisations
	\item DF Days Out
	\item Cheeky fundraising on the side
	\item Help Naomi with training stuff
\end{itemize}

\subsection{Events, Sustainability (Lily)}
\begin{itemize}
	\item Open bookings at previous events
	\item Info pack to be available at the time of booking
	\item Make videos
	\item Book venues before christmas
	\item Database of legit campsites and other venues, talking to districts, kinsfolk, etc.
	\item Make posts about fairer fare
	\item Make posts about cheap travel
	\item Google Docs folder for each event
	\item Events checklists for organisers
	\item Maintain events pack
	\item Not let teams use Facebook to organise
	\item Force co-ordinators to read events packs
	\item Evaluation forms after events
	\item Get reports from co-ordinators
	\item Get lists of people who are first-aid and MEST-UP trained early and pass them on
	\item DF Buddy system
	\item Support the Share
	\item Give Issy the chair
	\item Reviews of events on STW
\end{itemize}

\subsection{Chair (Imogen)}
\begin{itemize}
	\item Make business more transparent to the rest of DFs
	\item Non-specific-role-holding volunteers to feel really valued
	\item To have a really happy and functioning committee
	\item Be available for international-related queries
	\item Chase up Ellie Ferguson
	\item Don't die
	\item Help with Althing
	\item Get back to communications with committee quickly
	\item Don't get too stressed out
	\item Delegate
	\item Keep track of chasing people up
	\item Chase up non-committee as well as committee roles
\end{itemize}

\subsection{Shadow Treasurer (Nick)}
\begin{itemize}
	\item \emph{He said some things, check them}
	\item Money for WWOOFing
	\item Keep good track of receipts
	\item Be Ellie's communicative bitch
	\item Make pretty infographics
	\item Maybe talk to David about treasuring
	\item If it's not online, put it online
	\item Keep track of grant money
	\item Keep in contact with Richard Robertson (national Woodcraft treasurer)
\end{itemize}

\subsection{International Opportunities (Anna)}
\begin{itemize}
	\item Advertise to DFs about opportunities in interesting ways
	\item Get an international delegation to an event this year (maybe Spring Awakening)
	\item Get WWOOFing going again
	\item DF delegation to Europie (international youth forum thing)
	\item Make links with Syrian groups in the UK, and maybe organise talks
	\item Look for opportunities for younger DFs (16-17)
	\item Talk to Josh about Western Sahara stuff
	\item Advertise YOFest which is in Brussels and therefore awesome
	\item Learning about international things, not necessarily going there
	\item Be awesome on GC
\end{itemize}

\subsection{Projects (Joe F)}
\begin{itemize}
	\item Do loads and loads of evaluation
	\item Bring back an old project, or start a new one
	\item Support the Share
	\item Update all the project pages on STW
	\item Venturer involvement in projects
	\item Get PFAs approved by committee fast
	\item Keep talking with your hands
	\item Talk a bit louder --- project
\end{itemize}

\subsection{Everybody}
\begin{itemize}
	\item Be aware of core vs. extra things to do
	\item Git yo asses all up on Crabgrass
	\item Do quarterly reports
	\item Support each other
\end{itemize}

\section{Quotes}
\myquote[Louise Delmege]{Sometimes we have to choose between Scottish and Jewish people.}

\myquote[Lily Bowler]{We don't want to \emph{kill} Josh.}

\myquote{I feel like my thighs jiggled more than my bum there.}

\myquote[Anna Rathbone, dancing in excitement]{Don't get me started on public transport!}

\myquote[Naomi Wilkins]{I don't really know what's going on with General Council --- I should probably read that booklet.}

\myquote[Everybody on safeguarding]{Never gonna give it up; never gonna let it down; never gonna run around and desert it.}

\myquote[Jess Poyner]{Slide down my fucking body.}

\myquote[Saskia Neibig]{There is jargon, but don't be poetic.}

\myquote{Just keep poking Nick.}

\myquote[Saskia Neibig]{Just make up a qualification and work in a school with kids.}
\end{document}

