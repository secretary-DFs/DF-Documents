% !TEX TS-program = xelatex
% !TEX encoding = UTF-8

\documentclass[a4paper, 12pt]{article} % use larger type; default would be 10pt

\usepackage[british]{babel}
\usepackage{fontspec}
\defaultfontfeatures{Mapping=tex-text}
\setmainfont[Ligatures=TeX]{Linux Libertine}

\usepackage[scale=0.85]{geometry}
\usepackage[parfill]{parskip} % Activate to begin paragraphs with an empty line rather than an indent

\usepackage{graphicx} % support the \includegraphics command and options

\title{DF Democratic Procedures (Draft)}
\author{Joe MacMahon (DF Secretary)}
\date{\today}

\begin{document}
\maketitle

\section{Foreword}
Where the procedures specified in this document conflict with the DF Constitution, then the procedures in the Constitution shall prevail.  That is to say, this document may not override the Constitution.

\section{Business}
In all DF business (for example, at Althing, a Thing, or a mini-Thing), a system of consensus-based decision making should be used.  Participants are strongly encouraged to read ``The A-Z of Good Discussion'', which gives a good outline of the consensus system.  Recommendations for consensus are also covered in Appendix \ref{sec:consensus} of this document.

Additionally, at the beginning of business, the chair will briefly explain the principles and practicalities of this system, in particular the hand signals being used.  The chair may of course also deliver this explanation at later points in the discussion if, for example, new people have joined in or on return from a break.

No business may be undertaken without a quorum of members present, as outlined in the Constitution.

\section{Motions}
A motion is a specific instruction or statement agreed upon as reflecting the wishes or opinions of the DF Movement at large.  Motions are debated and voted on formally at Althing, and it is the job of the Secretary to prepare and publish an agenda of motions in advance of Althing.  All motions must have a proposer, who submits the motion for debate at Althing. If a motion is agreed upon, it is said to \emph{pass} or \emph{be carried}, and if not, it is said to \emph{fall}.

\subsection{Submission}
A motion may be submitted by any DF, DF Committee, or Regional Council.  Motions may be submitted at any point up to and including Althing, however the Secretary may set a deadline by which motions must be submitted in order to be included on the agenda.  Motions submitted after this deadline will only be discussed at Althing if there is enough time after all previously submitted motions.

\subsection{Discussion}
When it is a turn of a particular motion to be discussed, the chair will call for somebody to speak for the motion, usually the proposer.  If nobody wishes to speak for the motion, then it automatically falls and business moves on to the next agenda item.  If somebody is found to speak for the motion, the chair then immediately calls for somebody to speak against it.  If nobody is found to speak against it, then the motion is automatically carried.  If people are found to speak both for and against it, then these two people open the debate.  Once they have finished, the motion is discussed by everybody using the usual consensus system.

If a discussion concludes naturally, or appears to be going round in circles with no fresh ideas, or needs to end because of time constraints, the chair may take several more points and then move to a vote.

\subsection{Voting}
Voting on motions is usually done by raising of hands.  The chair will ask those in favour of the motion to raise their hands, and then those against, and finally any abstentions.  For motions which do not alter the Constitution to pass, a simple majority is required, i.e. 50\% + 1 vote.  Motions which do alter the Constitution require a 2/3rds majority vote to pass, i.e. 2/3rds + 1 vote.  When counting totals of votes cast, abstentions do not count, and quorum applies only to those votes cast.

\subsection{Amendments}
At any time up until voting on a motion, amendments may be submitted (by a DF, DF Committee, or Regional Council) which change an aspect of the motion.  Discussion and voting for amendments always takes place before discussion of the motion, and then the motion is discussed `as amended' or otherwise.

The format for discussing and voting on amendments is the same as for motions, with the exception that you cannot propose an amendment to an amendment, and all amendments are decided by a simple majority vote.

\subsection{Conflicting Motions/Amendments}

\section{Musings}

\section{Policy}

\section{Elections}

\section{Committees}

\appendix

\section{Consensus}
\label{sec:consensus}
In this section, some recommendations will be made regarding consensus-based decision making.  It is important, however, to note that this document recognises that this is a living system and therefore subject to change.  As such, it should be recognised that these are only recommendations, and if a particular business event wishes to adapt or modify them for its own purposes then it should feel free to do so.

\subsection{Chairing and speaking}
To participate in a discussion, you should indicate (to the chair) that you want to make a point, usually by raising your hand.  The chair will then call on you at a suitable time, when you should make your point in a clear and accessible manner.  You should make sure to listen to other people's points as well as making your own, and if your point has already been covered by another person, don't say it again.

Chairing should involve maintaining a list of people who wish to speak, firstly so that you can keep track of what order people raised their hands in, and secondly so that they don't have to keep their hands in the air until it is their turn to speak.  It is also good for the chair to periodically sum up the discussion and provide insight into proposals.  (Note that this does not mean adding more points or ideas to the discussion, merely summarising what has already been said.)

Business should begin naturally with one person chairing and introducing the topics for discussion, and they may choose to continue chairing the discussion throughout.  However it is important that the participants feel able to challenge the authority of the chair for a good reason (for example if the chair is biased or abusing their authority), and if they choose, appoint a new chair.  It is also good practice for different people to chair the discussion at different points so that more people can gain experience of chairing discussions.

\subsection{Important distinctions: Proposals, Points, and Direct Points}
\subsubsection{Proposals}
If you have a good idea of something to do, i.e. a specific action, you should make a proposal.  Proposals `jump the queue' and should be discussed immediately.  After a decision is made whether to carry out the proposal, the order of speakers should return to that which it was before the proposal was made.

For example:
\begin{itemize}
\item ``Why don't we set up a working group to look into this?''
\item ``I have a friend with a printing press, shall I ask her if we can use it?''
\item ``We should have this camp at XYZ campsite!''
\end{itemize}

\subsubsection{Direct Points}
If you have a direct response to something somebody has said, you may use the `direct point' hand signal (usually both hands raised), and jump the queue of speakers. Direct points must be used with caution and should be limited to a few sentences at most. They are points of information, and should never express an opinion.

For example:
\begin{itemize}
\item (In response to, ``How many people were at last Venturer Camp?'') ``734.''
\item (In response to, ``Maybe we should use Lockerbrook for this event.'') ``Lockerbrook is fully booked until November.''
\item (In response to, ``I think we should set up a DF Libarary.'') ``We are already in the process of doing exactly that, talk to Esther Price or Joe MacMahon if you want to get involved!''
\end{itemize}

\subsubsection{Normal points}

\subsection{Technical points}

\end{document}
