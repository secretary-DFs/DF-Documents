% !TEX TS-program = xelatex
% !TEX encoding = UTF-8

\documentclass[a4paper, 12pt]{article} % use larger type; default would be 10pt

\usepackage[british]{babel}
\usepackage{fontspec}
\defaultfontfeatures{Mapping=tex-text}
\setmainfont[Ligatures=TeX]{Linux Libertine}

\usepackage[scale=0.9]{geometry}
\usepackage[parfill]{parskip} % Activate to begin paragraphs with an empty line rather than an indent
\usepackage[hidelinks]{hyperref}

\usepackage{xr}
\externaldocument[const.]{../df_constitution}

\usepackage{graphicx} % support the \includegraphics command and options

\newcommand{\todo}{\emph{This section is not yet written.}}

\title{DF Democratic Procedures (Unfinished Draft)}
\author{Joe MacMahon (DF Secretary)}
\date{\today}

\begin{document}
\maketitle

\tableofcontents

\section{Foreword}
Where the procedures specified in this document conflict with the DF Constitution, then the procedures in the Constitution shall prevail.  That is to say, this document may not override the Constitution.

\section{Business}
In all DF business (for example, at Althing, a Thing, or a mini-Thing), a system of consensus-based decision making should be used.  Participants are strongly encouraged to read ``The A--Z of Good Discussion'', which gives a good outline of the consensus system.  Recommendations for consensus are also covered in \autoref{sec:consensus} of this document.

Additionally, at the beginning of business, the chair will briefly explain the principles and practicalities of this system, in particular the hand signals being used.  The chair may of course also deliver this explanation at later points in the discussion if, for example, new people have joined in or on return from a break.

\subsection{Quoracy}
No business may be undertaken without a quorum of members present, as outlined in Section \autoref{const.sec:quorum} of the Constitution.

\subsection{Expenses to Business Events}

\section{Motions}
A motion is a specific instruction or statement agreed upon as reflecting the wishes or opinions of the DF Movement at large.  Motions are debated and voted on formally at Althing, and it is the job of the Secretary to prepare and publish an agenda of motions in advance of Althing.  All motions must have a proposer, who submits the motion for debate at Althing. If a motion is agreed upon, it is said to \emph{pass} or \emph{be carried}, and if not, it is said to \emph{fall}.

\subsection{Submission}
A motion may be submitted by any DF, DF Committee, or Regional Council.  Motions may be submitted at any point up to and including Althing, however the Secretary may set a deadline by which motions must be submitted in order to be included on the agenda.  Motions submitted after this deadline will only be discussed at Althing if there is enough time after all previously submitted motions.

Motions should be `seconded' by a DF, DF Committee, or Regional Council.  The seconder of a motion does not have to be distinct from its proposer.

\subsection{Discussion}
When it is a turn of a particular motion to be discussed, the chair will call for somebody to speak for the motion, usually the proposer.  If nobody wishes to speak for the motion, then it automatically falls and business moves on to the next agenda item.  If somebody is found to speak for the motion, the chair then immediately calls for somebody to speak against it.  If nobody is found to speak against it, then the motion is automatically carried.  If people are found to speak both for and against it, then these two people open the debate.  Once they have finished, the motion is discussed by everybody using the usual consensus system.

If a discussion concludes naturally, or appears to be going round in circles with no fresh ideas, or needs to end because of time constraints, the chair may take several more points and then move to a vote.

\subsection{Voting}
\label{sec:motionvoting}
Participants may cast a vote for a motion, or against it, or they may abstain from voting.  Voting on motions is not usually done in secret, however the discussion as a whole may decide (by consensus or otherwise) to hold the vote in secret.  For motions which do not alter the Constitution to pass, a simple majority is required, i.e. 50\% + 1 vote.  Motions which do alter the Constitution require a 2/3rds majority vote to pass, i.e. 2/3rds + 1 vote.  When counting totals of votes cast, abstentions do not count, and quorum applies only to those votes cast.

\subsection{Amendments}
At any time up until voting on a motion, amendments may be submitted (by a DF, DF Committee, or Regional Council) which change an aspect of the motion.  Discussion and voting for amendments always takes place before discussion of the motion, and then the motion is discussed `as amended' or otherwise.

The format for discussing and voting on amendments is the same as for motions, with the following exceptions:
\begin{itemize}
\item You cannot propose an amendment to an amendment
\item The proposer of a motion may choose accept an amendment, in which case the amendment is applied automatically without discussion or a vote.
\item All amendments not accepted by the proposer of the motion are decided by a simple majority vote
\end{itemize}

\subsection{Conflicting Motions/Amendments}
\label{sec:conflicting}
Where two or more motions or amendments conflict, i.e. negate one another fully or in part, the meeting will decide by AV or Run-off Voting which of the possible alternatives they prefer, and subsequently whether to pass it at all.  For example, if there are three motions, A, B, and C, which negate each other, the participants might decide to prefer motion B over the others.  Then the other motions (A and C) are dropped, and the remaining motion is discussed in the normal way.

\section{Musings}
A musing consists of a thought or a question, formulated to inspire creative discussion.  A musing is less formal than a motion, and are discussed in the `open space' system, guidelines for which are to be found in \autoref{sec:openspace} of this document.  Musings have an open result; that is to say the actions to come out of a musing are not predefined.  It may be the case that a motion is formulated and proposed as a result of a musing, but equally a working group may be set up, or a project started by interested parties.

\section{Policy}
The term `policy' shall refer to all continuous or recurring effects put into place by passed motions at an Althing. This does not include modifications to the Constitution or to any document referred to by the Constitution.

Policy should last for 2 years by default, but should be able to be extended for another year at the discretion of Althing. The mechanism for this is as follows:

At each Althing, the Secretary shall propose a motion which causes each policy-making motion from the Althing two or more years previously to lapse. This motion should specifically list each of these motions so that they may be amended out (and thus continue to be in effect) at Althing's discretion.

Althing is advised that the criteria for letting a motion lapse should be that it is no longer relevant or current practice.  It should be noted that if a lapsing motion should be amended in some way, it is best to let it lapse and re-propose an amended version of it.

\section{Elections and Roles}
Where the election for a particular role is held depends on the motion which created the role, however, most elections take place at Althing.  In all elections, a virtual candidate called `RON' (standing for Re-Open Nominations) shall be electable.  In most cases, if RON is elected, a request for further nominations of candidates is put out, and the election is postponed until the next available business event.  The exception to this, as detailed in section \autoref{const.sec:treasurernoconfidence} of the Constitution, is the role of Treasurer.

\subsection{Nominations}
Before an election is held, nominations for candidates may be made to the Secretary.  The Secretary will establish if each nomination is accepted by its nominee, and if so their name will be included in the final list of candidates.

The virtual candidate `Re-Open Nominations' is automatically nominated in all elections.

\subsection{Hustings and Voting}
At the beginning of the election, the Secretary will close nominations and invite all the candidates to give a short speech, known as a \emph{hustings}.  The purpose of the hustings is to allow candidates to inform and persuade voters of their electability.  A candidate does not have to give a hustings, but if they do it is recommended that they take questions at the end of it.

All elections to single roles on DF Committee and Regional Council shall take place by AV.  Elections where there are multiple positions available to the same pool of candidates will take place by STV.  This includes, for example, the special case where one Lay Member stands down mid-term, and therefore both Lay Members are elected in the same year.

\section{Voting Practicalities}
Voting is usually required to pass a motion, and to elect a candidate.  As detailed in \autoref{sec:motionvoting}, voting on motions is usually done by a simple show of hands for votes for, votes against and abstentions.  However, elections for roles and certain special cases for motions (conflicting motions or amendments, see \autoref{sec:conflicting}) use voting systems that require some explanation.

\subsection{Secret Ballots}
Elections are usually held in secret, with votes expressed written down anonymously on ballot papers.  These votes are then collected and counted in public, and the result announced to the electorate.  When voting on motions is held in secret, a similar process may be used, or possibly for convenience, members of the discussion may be asked to close their eyes and vote normally with hands.

\subsection{Tellers}
This of course raises the question of who does the counting.  These people are known as \emph{tellers} and are chosen by consensus from a discussion.  It is their job to make sure voting is carried out and votes are counted correctly.  To make sure tellers are (relatively) neutral, they must surrender their right to vote in the election they are administering.

\subsection{Voting Systems}
The type of voting system used in an election will depend on the details of the election.  Usually, AV is used for elections with one position available, STV for elections with multiple positions and Run-off Voting used as an alternative to AV, when AV would be too time consuming.

\subsubsection{Alternative Vote (AV)}
\label{sec:av}
Under the Alternative Vote system, a single candidate is elected from a pool of candidates.  Each voter expresses a ranked list of preferences on their ballot paper, in the order in which they would prefer those candidates to win.  Then, once collected, the following process is used to establish a winner of the election:

\begin{enumerate}
\item Votes are counted based on first preferences, discarding spoilt votes.
\item \label{item:candidatemajority} If a candidate has a 50\% + 1 majority, they are elected.
\item Otherwise, the candidate with the least votes is eliminated, and their votes are redistributed based on the next preference.
\item Return to step \ref{item:candidatemajority}.
\item If there is a tie, the election is held again with only those two candidates eligible for votes.
\item If there is still a tie, a coin-toss may decide the vote, or the two tied candidates may agree to share the position in some way.
\end{enumerate}

\subsubsection{Run-off Voting}
\todo
\subsubsection{Single Transferrable Vote (STV)}
\todo

\appendix

\section{Open Space}
\label{sec:openspace}
\todo

\section{Consensus}
\label{sec:consensus}
In this section, some recommendations will be made regarding consensus-based decision making.  It is important, however, to note that this document recognises that this is a living system and therefore subject to change.  As such, it should be recognised that these are only recommendations, and if a particular business event wishes to adapt or modify them for its own purposes then it should feel free to do so.

\subsection{Chairing and speaking}
To participate in a discussion, you should indicate (to the chair) that you want to make a point, usually by raising your hand.  The chair will then call on you at a suitable time, when you should make your point in a clear and accessible manner.  You should make sure to listen to other people's points as well as making your own, and if your point has already been covered by another person, don't say it again.

Chairing should involve maintaining a list of people who wish to speak, firstly so that you can keep track of what order people raised their hands in, and secondly so that they don't have to keep their hands in the air until it is their turn to speak.  It is also good for the chair to periodically sum up the discussion and provide insight into proposals.  (Note that this does not mean adding more points or ideas to the discussion, merely summarising what has already been said.)

Business should begin naturally with one person chairing and introducing the topics for discussion, and they may choose to continue chairing the discussion throughout.  However it is important that the participants feel able to challenge the authority of the chair for a good reason (for example if the chair is biased or abusing their authority), and if they choose, appoint a new chair.  It is also good practice for different people to chair the discussion at different points so that more people can gain experience of chairing discussions.

\subsection{Important distinctions: Proposals, Points, and Direct Points}
\subsubsection{Proposals}
If you have a good idea of something to do, i.e. a specific action, you should make a proposal.  Proposals `jump the queue' and should be discussed immediately.  After a decision is made whether to carry out the proposal, the order of speakers should return to that which it was before the proposal was made.

For example:
\begin{itemize}
\item ``Why don't we set up a working group to look into this?''
\item ``I have a friend with a printing press, shall I ask her if we can use it?''
\item ``We should have this camp at XYZ campsite!''
\end{itemize}

\subsubsection{Direct Points}
If you have a direct response to something somebody has said, you may use the `direct point' hand signal (usually both hands raised), and jump the queue of speakers. Direct points must be used with caution and should be limited to a few sentences at most. They are points of information, and should never express an opinion.

For example:
\begin{itemize}
\item (In response to, ``How many people were at last Venturer Camp?'') ``734.''
\item (In response to, ``Maybe we should use Lockerbrook for this event.'') ``Lockerbrook is fully booked until November.''
\item (In response to, ``I think we should set up a DF Libarary.'') ``We are already in the process of doing exactly that, talk to Esther Price or Joe MacMahon if you want to get involved!''
\end{itemize}

\subsubsection{Normal points}
\todo

\subsection{Technical points}
\todo

\end{document}
